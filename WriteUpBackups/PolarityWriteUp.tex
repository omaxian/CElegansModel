\documentclass[11pt]{article}
\usepackage{caption}
\captionsetup[table]{font={stretch=1.0,small}}     %% change 1.2 as you like
\captionsetup[figure]{font={stretch=1.0,small}}    %% change 1.2 as you like
\linespread{1.5} 
\usepackage{graphicx,epstopdf,subfigure,mathtools,mathrsfs, arydshln, amsmath, amssymb} 
\usepackage[font=small,labelfont=bf]{caption}
\usepackage{float}
\usepackage[shortlabels]{enumitem}
\usepackage{authblk}
\usepackage[title]{appendix}
\PassOptionsToPackage{usenames,dvipsnames}{xcolor}
\usepackage[usenames,dvipsnames]{xcolor}
\usepackage[margin=1in]{geometry}
\usepackage[normalem]{ulem}

\usepackage[round]{natbib}

\usepackage{amsfonts}
\usepackage{hyperref}
\hypersetup{
    colorlinks=false,
    pdfborder={0 0 0},
}
\newcommand{\new}[1]{\color{blue}#1\normalcolor}
\newcommand{\red}[1]{\color{red}#1\normalcolor}
\newcommand{\delete}[1]{}
\newcommand{\change}[1]{\color{black}#1\normalcolor}
\newcommand{\rev}[1]{\color{black}#1\normalcolor}

% VECTOR AND MATRIX NOTATION
\newcommand{\V}[1]{\boldsymbol{#1}}                 % vector notation
\newcommand{\M}[1]{\boldsymbol{#1}}
\newcommand{\Lop}[1]{\boldsymbol {\mathcal{#1}}}
\global\long\def\Ac{A_\text{cyto}}
\global\long\def\Pc{P_\text{cyto}}
\newcommand{\CDC}[1]{#1_{\text{c}}}
\newcommand{\6}[1]{#1_{\text{6}}}
\newcommand{\3}[1]{#1_{\text{3}}}
\newcommand{\CHIN}[1]{#1_{\text{ch}}}
\global\long\def\kon{k^\text{on}}
\global\long\def\koff{k^\text{off}}
\global\long\def\kf{k^+}
\newcommand{\Tot}[1]{#1^\text{(Tot)}}
\newcommand{\Sat}[1]{#1^\text{(Sat)}}
\global\long\def\Dt{\partial_t}
\global\long\def\Dthat{\partial_{\hat{t}}}
\global\long\def\Dx{\partial_x}
\global\long\def\Dxhat{\partial_{\hat{x}}}
\global\long\def\MChinC{P_\text{cyto}}
\global\long\def\MChin{P_1}
\global\long\def\PChin{P_n}
\global\long\def\MAC{A_\text{cyto}}
\global\long\def\MA{A_1}
\global\long\def\PA{A_n}
\global\long\def\CDCy{C_\text{cyto}}
\global\long\def\CD{C}
\global\long\def\kp{k^\text{p}}
\global\long\def\kdp{k^\text{dp}}
\global\long\def\kI{k^\text{I}}
\global\long\def\kE{k^\text{E}}
\newcommand{\A}[1]{#1_A}
\newcommand{\Chin}[1]{#1_P}
\newcommand{\C}[1]{#1_C}
\global\long\def\DhatA{\hat{D}_A}
\global\long\def\KhatonA{\hat{K}^\text{on}_A}
\global\long\def\Khatoff{\hat{K}^\text{off}}
\global\long\def\KhatoffA{\hat{K}^\text{off}_A}
\global\long\def\KhatfA{\hat{K}^\text{f}_A}
\global\long\def\KhatpA{\hat{K}^\text{p}_A}
\global\long\def\KhatpAP{\hat{K}^\text{p}_\text{AP}}
\global\long\def\KhatdpA{\hat{K}^\text{dp}_A}
\global\long\def\DhatP{\hat{D}_P}
\global\long\def\DhatK{\hat{D}_K}
\global\long\def\KhatonP{\hat{K}^\text{on}_P}
\global\long\def\KhatonM{\hat{K}^\text{on}_M}
\global\long\def\KhatoffP{\hat{K}^\text{off}_P}
\global\long\def\KhatoffM{\hat{K}^\text{off}_M}
\global\long\def\KhatoffK{\hat{K}^\text{off}_K}
\global\long\def\KhatfP{\hat{K}^\text{+}_P}
\global\long\def\KhatpP{\hat{K}^\text{p}_P}
\global\long\def\KhatdpP{\hat{K}^\text{dp}_P}
\global\long\def\DhatC{\hat{D}_C}
\global\long\def\KhatonC{\hat{K}^\text{on}_C}
\global\long\def\KhatoffC{\hat{K}^\text{off}_C}
\newcommand{\My}[1]{#1_M}
\newcommand{\R}[1]{#1_R}

\title{CDC-42 encodes dynamically stable asymmetries in the \emph{C.\ elegans} zygote via an incoherent feed-forward loop}
%\title{Mathematical appendix: \\ Oligomerization and feedback on membrane recruitment stabilize PAR-3 asymmetries in \emph{C.\ elegans} zygotes}
\author{Ondrej Maxian, Cassandra Azeredo-Tseng, $\dots$, and Edwin Munro \vspace{-0.75 cm}}

\begin{document}
\maketitle

\section*{Introduction}
Cell polarity is essential for many aspects of organismal development and physiology, including stem cell dynamics, directional cell migration, and asymmetric cell division \citep{dewey2015cell, goldstein2007proteins, ierushalmi2021cytoskeletal, maitre2016asymmetric}. For most cells, the first step in polarization involves a symmetry-breaking response to a transient localized cue that creates asymmetric distributions of specific molecules or molecular activities. The mechanisms that underlie cellular symmetry-breaking have been extensively explored \citep{li2010symmetry}, but mechanisms that maintain polarity as a dynamically stable state with a fixed boundary position have only recently come under scrutiny \citep{sailer2015dynamic, gross2019guiding}.

On a large scale, a cell's polarity state is encoded by asymmetric distributions of protein molecules, which are shaped by smaller-scale processes like binding, diffusion, and active transport. In \emph{C. elegans}, polarity is encoded by the distribution of two distinct groups of (highly conserved) polarity proteins: anterior PARs (aPARs), which include the scaffold PAR-3, adaptor PAR-6, atypical kinase PKC-3, and GTPase CDC-42, and posterior PARs (pPARs), which include the RING-domain containing protein PAR-2 \citep{bland2023optimized}, kinase PAR-1, tumor suppressor LGL-1, and CDC-42 GAP CHIN-1 \citep{lang2017proteins}. 

Wild-type embryos polarize in two distinct phases termed ``establishment'' and ``maintenance'' \citep{cuenca2003polarization}. The mechanisms that underlie symmetry-breaking during polarity establishment have been well-studied \citep{cowan2007acto, munro2009cellular}.  Near the end of meiosis, PAR-3, PAR-6 and PKC-3 are uniformly distributed at the cortex, where they prevent cortical association of PAR-1, PAR-2 and LGL-1 \citep{schonegg2006cdc, others}.  One mode of symmetry-breaking involves the local inhibition of actomyosin contractility near the sperm MTOC, which triggers cortical flows that transport F-actin, myosin II, PAR-3/PAR-6/PKC-3, and other cortically associated factors towards the anterior pole, resulting in their mutual restriction to an anterior cap \citep{munro2004cortical, rodriguez2017apkc}. Posterior depletion of PAR-3/PAR-6/PKC-3 then allows PAR-1, PAR-2 and LGL-1 to associate with a complementary posterior domain.  Actomyosin contractility and cortical flow during polarity establishment require the small GTPase Rho-1 \citep{schonegg2006cdc, motegi2006sequential}, and in fact the sperm cue acts via Aurora A kinase (AIR-1) to locally inhibit Rho through the Rho GEF Ect-2 \citep{motegi2006sequential, tse2012rhoa, longhini2022aurora, kapoor2019centrosome}.  A second mode of symmetry-breaking has also been described in which sperm astral microtubules promote local association of Par-2 with the posterior cortex \citep{motegi2011microtubules}.

While the mechanisms that break symmetry are by now well understood, it remains unclear what \emph{stops} symmetry breaking. If the system is programmed to amplify an initial polarizing cue, what sets the limit of amplification? The by now standard mechanism is that anterior (CDC-42/PAR-3/PAR-6/PKC-3) and posterior proteins (PAR-1/PAR-2/LGL-1) form a bistable reaction diffusion system by competing for residence at the cortex/plasma membrane such that one or the other but not both win locally. Theoretical studies suggest that in principle such a mechanism could stabilize the AP boundary after establishment phase cortical flows cease, or in embryos that lack cortical actin or myosin \citep{mori2008wave, dawes20113, goehring2011polarization, lang2023oligomerization}. These mechanisms by themselves cannot, however, shift the boundary on realistic timescales \citep{lang2023oligomerization}, which is why flow is required.

In the presence of flow, it is still unknown what sets a limit to boundary progression. Recent work showed that, assuming an intrinsic self limit to contractility, the boundary can be pinned, with the position set by total amount of each PAR protein \citep{gross2019guiding, goehring2011polarization}. But the self-limiting nature of the flow remains a mystery. In wild type embryos, myosin accumulates at the anterior in a cap that takes up about 50\% of the embryo, and myosin flows from the posterior half of the embryo into the anterior cap. Despite the flow of myosin and the consequent A/P asymmetry in contractility, the cap maintains a fixed size (it does not contract), and the flow profile in the anterior is roughly zero \citep{sailer2015dynamic}. Moreover, ectopic accumulation of Myosin II during maintenance in PAR-2 mutants is associated with abnormal posterior-directed cortical flows and rapid redistribution of Par proteins \citep{munro2004cortical}, suggesting that the distribution of myosin, in addition to the PAR proteins, guides the steady state.

While we will show that the self-limiting nature of contractility applies in both establishment and maintenance phase, our focus in particular will be on maintenance, which requires CDC-42 to maintain asymmetries set up during establishment \citep{kay2001cdc, gotta2001cdc,aceto2006interaction, schonegg2006cdc, motegi2006sequential}. GFP-tagged CDC-42 becomes anteriorly enriched during polarity establishment, and studies with a GFP-tagged biosensor suggest that the active (GTP-bound) form of CDC-42 may be similarly enriched during maintenance phase \citep{kumfer2010cgef}.  Binding of CDC-42 to the conserved semi-Crib domain of PAR-6 is required for cortical association of PAR-6/PKC-3 during maintenance \citep{aceto2006interaction}, and PKC-3 is in an active state only when bound to CDC-42 \citep{sailer2015dynamic,lang2017proteins,rodriguez2017apkc}. CDC-42 also acts through MRCK-1, a C. elegans othologue of the mammalian Myotonia Dystrophy-related CDC-42-binding kinase MRCK, to promote asymmetric cortical recruitment of myosin II \citep{kumfer2010cgef}.  There is also evidence that CDC-42 promotes asymmetric enrichment of F-actin during maintenance phase (REF?), but the underlying mechanism remains poorly understood.  More generally, it remains unclear how these distinct outputs of CDC-42 are integrated to dynamically stabilize the AP boundary at a fixed axial position and limit contractility. 

In this study, we combine experiments and theoretical modeling to show that boundary progression can be stalled when distinct outputs of CDC-42 have different effects on contractility. By extending both establishment and maintenance phases, we show that the boundary position is in fact stable, regardless of the cell cycle phase. We then demonstrate, through experiments with a temperature sensitive ECT-2 gene \citep{zonies2010symmetry}, that the steady state observed at the end of maintenance phase can be reproduced even when establishment fails through a maintenance-phase rescue process, revealing that maintenance phase biochemistry encodes a dynamically stable attractive state. To understand how the PAR protein circuit (which involves CDC-42) interacts with myosin contractility to limit the extent of contractility, we introduce a continuum model. Without a self-limiting character to the flow, the boundary either contracts all the way to the end of the embryo, or slowly moves to the embryo interior over a timescale an order of magnitude longer than maintenance phase. To find the missing model component, we return to experiments, which demonstrate that branched actin in the anterior acts to reduce tension and prevent excessive contraction, similar to behavior observed previously in other systems \citep{muresan2022f, yang2012arp2}. Our model reveals that adding a threshold of CDC-42, above which branched actin is produced, is sufficient to reproduce the initial and latter stages of rescue, with the caveat that additional assumptions about branched actin are required to reproduce all experimental observations.


\section*{Results}

\subsection*{Establishment and maintenance phase contractility are governed by unique biochemical systems}

As a first step to examining the role of CDC-42 in regulating contractility, we sought to distinguish the phases of the cell cycle in terms of the upstream protein which ultimately regulates myosin. We began by imaging wild-type embryos expressing non-muscle myosin II fused to GFP (NMY-II::GFP), as well as embryos expressing PAR-6::GFP. At pseudo-cleavage and prior to the onset of centration (Fig.\ 1a), myosin exhibited pulsatile dynamics, with large foci appearing and disappearing over time, and PAR-6 was enriched in clusters on the anterior cortex (Fig.\ 1b). By nuclear envelope breakdown, both myosin and PAR-6 had transitioned to diffuse, smaller clusters enriched on the anterior cortex. Pulsatile myosin activity during establishment phase has previously been shown to involve rho-mediated contractility \citep{michaux2018excitable, michaud2022versatile, yao2022modulating}, while the diffuse myosin clusters have been previously linked to CDC-42 \citep{motegi2006sequential}. Likewise, previous reports \citep{motegi2006sequential, rodriguez2017apkc} have shown that the diffuse PAR-6 clusters are correlated with binding to CDC-42, which activates PKC-3, while the punctate clusters are linked to PAR-3. We therefore conclude that Rho-1 mediates contractility during polarity establishment phase, while CDC-42 controls both contractility and PAR polarity during maintenance phase. The coupling of biochemistry and contractility through CDC-42 makes maintenance unique relative to establishment.

To understand the transition from establishment to maintenance, we depleted embryos of PRI-1 via RNAi. Depletion of PRI-1 prevented DNA replication and stalled cells in late establishment phase (Fig.\ 1c). In particular, we found a typical delay time of about eight minutes from pronuclear meeting to centration onset, relative to wild type embryos (Fig.\ 1d). The delay in centration onset also correlated with a delay in pseudo-cleavage relaxation (middle kymographs in Fig.\ 1c), and prolonged flows from the posterior into the psuedo-cleavage region (rightmost kymographs in Fig.\ 1c). Regardless of the embryo treatment, we found a strong correlation between three events: centration onset, psuedo-cleavage relaxation, and the myosin transition from punctate to diffuse. We conclude that these events mark the transition from establishment to maintenance, and especially to the regime of CDC-42-mediated contractility. While this regime is our interest here, it is noteworthy that embryos in late establishment phase (especially PRI-1 depleted embryos) exhibit a stable A/P myosin asymmetry characterized by persistent anterior-directed flow, similar to previous observations for maintenance phase. We return to this point in the discussion.


\subsection*{The steady state of maintenance phase is a stable boundary with persistent anterior-directed flow}
A recent examination of polarity establishment linked the end of establishment phase with a vanishing flow profile and a stable A/P asymmetry coming from mutual exclusion of PAR proteins \citep{gross2019guiding}. Yet, embryos in late-maintenance phase exhibit a steady pattern of anterior-directed cortical flow with an asymmetric myosin profile \citep[Fig.~2]{sailer2015dynamic}. We consequently wondered which of the two pictures represents the true ``steady state'' of maintenance phase.  

To probe whether the previously-observed myosin intensity and patterns of flow in late maintenance phase are truly steady states, we extended maintenance phase by depleting wild-type embryos of the cell cycle regulator CDK-1. Marking maintenance phase as the period between pseudocleavage relaxation and anaphase onset, we obtained a window of roughly 8--10 minutes per cell, in which we characterized the myosin intensity and flow speeds. Later stages of the extended maintenance phase showed a relatively stable position of the myosin boundary (it in fact expands towards the posterior slightly) and a persistent anterior-directed flow which tends to achieve a maximum (magnitude) just posterior of the peak myosin location (Fig.\ 2A). However, because the myosin intensity and flow speeds in CDK-1 (RNAi) never resemble those of wild-type, it is clear that interfering with CDK-1 affects more than just the cell cycle, and it is impossible to say concretely that the specific flow and myosin profiles in late maintenance are ``steady.'' Nevertheless, we can conclude that an asymmetric myosin profile and persistent anterior-directed flow do not lead to further contraction of the boundary when maintenance is extended.

\subsection*{The maintenance phase steady state is unique and attractive in the presence of MRCK}
We next wanted to determine if the boundary position in maintenance phase is unique, or if it is a consequence of the dynamics of establishment. Previous studies in \emph{ect-2} mutants (which fail to establish polarity due to lack of rho activity) have reported a delayed symmetry breaking \citep{zonies2010symmetry, tse2012rhoa}, but have not directly explained this as a maintenance-phase phenomenon, nor studied how the boundary position and flow profile change relative to normal maintenance phase. 

To probe these dynamics further, we systematically imaged symmetry breaking in temperature-sensitive \emph{ect-2} mutants, marking the beginning of maintenance phase via the transition of myosin from large to small clusters (since there is no pseudocleavage), and the end of maintenance phase as the onset of embryo rotation prior to cytokinesis. As described in previous reports \citep{zonies2010symmetry}, embryos exhibited symmetry breaking that we refer to as ``maintenance phase rescue,'' since it occurred only when myosin was distributed in diffuse clusters. During rescue, myosin and anterior PAR proteins segregated into an anterior domain of the same size as in wild-type embryos. At the end of rescue, the pattern of flow was characterized by a stronger flow in the posterior half of the cell into a stall point at the edge of the anterior cap, as in wild type embryos (Fig.\ 2B). Embryos depleted of the kinase MRCK, which acts downstream of CDC-42 to activate myosin, failed to rescue polarity in maintenance phase, as neither the anterior domain nor myosin domain contracted towards the anterior (Fig.\ 2B).

Having characterized maintenance phase dynamics when establishment phase is either totally effective or totally ineffective, we wondered if having partial polarity establishment would lead to partial polarity rescue. To accomplish this, we depleted embryos of NOP-1, a protein which partially mediates contractility by activating rho \citep{tse2012rhoa}. Exposing embryos to varying degrees of NOP-1 (RNAi) led to varying boundary positions at the end of establishment phase. Upon the transition to maintenance phase, cortical flows were initiated, and the boundaries in each of the cells converged to the same approximate position at roughly 60\% embryo length (Fig.\ 2C). By contrast, in the absence of MRCK, the A/P boundary stayed in the same position throughout maintenance phase, and no cortical flows were initiated. We conclude that maintenance phase encodes an attractive steady state with a unique boundary position and flow pattern. When establishment is defective, CDC-42/MRCK-mediated cortical flows during maintenance phase rescue the correct boundary position.

\subsection*{Existing models cannot explain the dynamics of maintenance phase rescue }
We now explore whether a mathematical model, informed by existing knowledge of maintenance phase, could explain its tendency to correct errors from establishment. At first glance, it appears that the phenomenon of maintenance-phase rescue could be a consequence of an instability in the underlying myosin dynamics. In the SI, we consider this possibility by performing linear stability analysis of a simple myosin model informed by our experimental data \citep{mayer2010anisotropies, bois2011pattern}. The stability analysis shows that the dynamics are unstable when the flow carries myosin molecules a distance larger than the hydrodynamic lengthscale, which is the typical length a local disturbance propagates through flows \citep{mayer2010anisotropies}. For realistic lifetimes on the order 5--20 s, the minimum flow speed to generate instability is about 40 $\mu$m/min, which is much faster than ever observed in maintenance phase. Thus the dynamics of maintenance phase rescue are not due to myosin instabilities.

Because the dynamics of myosin alone are insufficient to generate instability, PAR proteins must be essential for rescue to occur, as was previously shown in the case of PAR-2 \citep{zonies2010symmetry}. We consequently introduce a model of maintenance phase biochemistry based on previously-characterized interactions (Fig.\ 3A), \citep{lang2017proteins}. On the anterior side, we have three distinct protein species: PAR-3, CDC-42, and PAR-6/PKC-3, each of which has a separate function. The posterior PARs (PAR-2, PAR-1, and CHIN-1) can be lumped into one species (denoted by $P$), which antagonizes both PAR-3 and CDC-42. A schematic diagram is given in Fig.\ 3A, and equations which describe the circuit are given in the SI. In brief, PAR-3 is locally bistable dynamics because of oligomerization and positive feedback \cite{lang2023oligomerization}. The enriched zone of PAR-3 sets up a gradient of PAR-6/PKC-3, which inhibits all of the posterior PAR proteins. The posterior PARs then inhibit CDC-42 (directly through CHIN-1 and indirectly through PAR-2) \citep{munro2004cortical, sailer2015dynamic}. 

In the absence of contractility, our model shows that the maintenance phase circuit has a unique boundary position, where the PAR-3 boundary sits at about 50\% embryo length (Fig.\ 3B). Under normal circumstances, where the boundary begins at 50\% embryo length (the end of establishment phase), the boundary rapidly adjusts to its unique position in under ten minutes. By contrast, under rescue conditions, where there is local loading of pPARs in the posterior-most 10\% of the embryo, the boundary slowly corrects, taking about an hour to reach steady state. The steady state here is reached through a wave-pinning mechanism \citep{mori2008wave, goehring2011polarization}, where the boundary stops moving when reaction and diffusion fluxes (both of which depend on the cytoplasmic protein concentration) come into balance. The change in protein concentration occurs through unbinding, which for most proteins occurs on the timescale of hundreds of seconds \citep{robin2014single}, thus explaining the slow dynamics. Our model therefore confirms the experimental result that rescue is impossible without flows.

To model contractility, we make CDC-42 a promoter of myosin at the cortex, and cortical tension directly proportional to myosin (blue parts in Fig.\ 3A, see SI for equations). Flow occurs in the direction of gradients in cortical tension, and all proteins on the cortex are advected with the local cortical velocity \citep{illukkumbura2023design}. With contractility, simulations under rescue conditions reproduce the initial stages of rescue (Fig.\ 3C), where an initially peaked profile of pPARs invades the anterior domain, concentrating aPARs in the middle and thereby increasing the concentration of pPARs in the posterior. As a result of this, CDC-42 gets inhibited in the posterior, which gives a gradient of myosin from posterior to anterior. The gradient of myosin generates a flow which further compacts the anterior domain. The timescale of this compaction is much faster than without flows, and indeed occurs on a timescale of minutes and not hours.

While the model reproduces the initial stages of the rescue process, there is no mechanism in it to halt the advancing myosin front. Indeed, without flows the reaction-diffusion mechanism relied on aPARs becoming enriched in the anterior to eventually out-compete pPARs and stall the boundary. But when enrichment of aPARs also enriches myosin, flow speeds also increase, and the additional advective flux of pPARs overwhelms the increasing reactive flux, resulting in an ever-contracting boundary with ever-increasing flows and pPARs becoming uniformly enriched.

In order to properly reproduce rescue, the model suggests a need for a local inhibitor of contractility. If an additional agent can inhibit contractility without inhibiting other aPAR proteins, the reactive flux and advective flux would decouple, and the reaction-diffusion mechanism can stall the boundary. We return next to experiments to find such an inhibitor.

\subsection*{Branched actin prevents hypercontractility and is necessary for successful rescue}


Our modeling so far identified two regimes of behavior, depending on the sensitivity of myosin to the CDC-42 concentration. Roughly speaking, if CDC-42 promotes myosin at a rate much smaller than the basal rate (0 in Fig.\ \ref{fig:BoundaryDiff}), the cytoplasmic dynamics are sufficient to stop the pPARs from invading too far into the anterior domain. But this model has to be discarded because the dynamics occur over unrealistically long timescales. To match the speed of rescue, CDC-42 has to promote myosin at a rate much larger than the basal rate, which leads to fast flows and (in this model) a rapid concentration of the anterior domain into a peaked profile at the anterior pole (Fig.\ \ref{fig:RescueNoBA}). Thus, we are missing an important interaction that slows down the advancing myosin peak, rapidly dropping the flow speed and allowing cytoplasmic depletion to pin the boundary. This interaction is the focus of this section.

In Fig.\ \ref{fig:Ect2VsWT} we show the myosin intensity and flow speed over 30 second intervals of maintenance phase rescue, and compare the result to the steady myosin intensities and flow speeds observed in wild-type embryos \citep{sailer2015dynamic}. The myosin peak initially grows in size advances rapidly towards the anterior, but after about three minutes it stops growing and becomes pinned at about 40\% embryo length. The pinning of the myosin peak corresponds to a decrease in the maximum flow speed from about 4.5 $\mu$m/min (from 2:00 to 3:30) to 2 $\mu$m/min in the last 30 seconds of maintenance phase (3:30 to 4:00, see the inset of Fig.\ \ref{fig:Ect2VsWT}). In the last 30 seconds of maintenance phase, the flow profile in \emph{ts ect-2} embryos is roughly the same as that observed during late maintenance phase in wild-type embryos. The myosin intensity appears quantitatively different, but has the same fundamental shape: there is a gradient of myosin in both the anterior and posterior, with a maximum occurring at around 40\% of the embryo length. These results establish that the biochemical circuit governing maintenance phase can amplify small residual asymmetries from establishment phase, and that the resulting polarized state is independent of what comes before maintenance. 

We hypothesize that branched actin might stop the anterior cap from becoming hyper-contractile. To test this hypothesis, we deplete wild type embryos of arx-2, a protein which activates the arp 2/3 complex and the assembly of branched actin. Figure\ \ref{fig:Arp23Myosin} shows the average myosin intensity and flow profiles over the last three minutes of maintenance phase in wild-type (red) and arx-2 (RNAi) treated embryos (blue). While the wild-type intensity seems to be roughly stalled at a boundary position 40\% of the embryo length, the arx-2 (RNAi) embryos demonstrate hyper-contractility, with a myosin peak that shifts to the anterior (and increases) over time.

The flow measurements in \emph{arx-2} (RNAi) embryos are chaotic and noisy because the hyper-contractile anterior cap is not always positioned directly on the anterior pole, so flows can be generated off the A/P axis. In Fig.\ \ref{fig:Arp23Myosin}, we show the average flow speeds in embryos where the A/P flow dominates other directions (6 of the 12 embryos). These flows have a maximum speed around 2.5--4 $\mu$m/min. Unlike the wild-type profile, whose point of maximum flow and stall point are roughly constant in time, the \emph{arx-2} (RNAi) flow profiles show no stall point, are larger in magnitude, and have a local maximum whch is positioned far to the anterior of the wild-type. 

We hypothesized that the chaotic and noisy flow speeds in \emph{arx-2} (RNAi) embryos were a result of the smaller anterior cap being pulled and pushed around by forces from the mitotic spindle. Because of this, we applied nocodazole, a drug which depolymerizes microtubules, and observed the resulting dynamics. Interestingly, nocodazole treatment of wild type embryos appears to remove the posterior-directed flow earlier in early maintenance phase (Fig.\ \ref{fig:Arp23Myosin}) \citep{sailer2015dynamic}. In late maintenance, however, the dynamics are relatively unchanged with nocodazole treatment, with a peak in the myosin concentration around 40\% embryo length and a maximum flow speed of 2 $\mu$m/min at 60\% embryo length (blue lines in Fig.\ \ref{fig:Arp23Noc}).

In \emph{arx-2} (RNAi) embryos, nocodazole treatment significantly improved the patterns of flow, confining them to the A/P axis and allowing us to extract meaningful information. Similar to the case without nocodazole, embryos treated with arx-2 (RNAi) and nocodazole showed a hypercontractile anterior myosin cap (Fig.\ \ref{fig:Arp23Noc}), with the cap continuing to contract over time to the very edge of the anterior domain. The contraction of the cap corresponded with marked increases in flow speed during maintenance, so that flows went from 1.5 $\mu$m/min during the third minute before anaphase onset to 3--4 $\mu$m/min, with the point of maximum flow occuring at 40\% embryo length instead of 60\% (Fig.\ \ref{fig:Arp23Noc}). Thus, regardless of the presence of microtubules, depletion of arx-2 by RNAi leads to a hypercontractile state with a larger velocity and a cap which tends to fall of the edge of the embryo, similar to what we observed in simulations in Fig.\ \ref{fig:RescueNoBA}. We now propose a model for how branched actin could inhibit contractility and lead to the formation of a stable boundary.

\subsection*{Incorporating branched actin into existing models correctly reproduces rescue dynamics}
To model branched actin, we work off the following two important observations from our maintenance-phase rescue experiments: first, the pinning of the myosin domain and drop in velocity happen quite abruptly, and second, the rapid drop in velocity is \emph{not} accompanied by strong changes in the myosin concentration profile (Fig.\ \ref{fig:Ect2VsWT}). These observations together suggest that there is an ultra-sensitive dependence of contractility (not myosin concentration) on branched actin. As rescue progresses, branched actin becomes active on the anterior, providing a ``brake'' which slows down contractility and progression of the boundary.

To translate our hypothesis into our model, we make the following assumptions, 
\begin{enumerate}
\item To activate branched actin, we assume a threshold of CDC-42 above which branched actin is created. The threshold is reached roughly when the A/P boundary reaches the middle of the cell.
\item Branched actin assembly is autocatalytic, since new branches become sites for additional branches.
\item Branched actin unbinds from the cortex with characteristic rate $\koff_R$ (about 8 s lifetime).
\item While branched actin does not diffuse in the cortex, polymerization of actin leads to a spreading out of the branched network, and so we include nonzero diffusivity. 
\item Branched actin inhibits contractility. 
\end{enumerate}
The first four assumptions lead to the advection-diffusion-reaction equation for branched actin given in the SI. For the fifth assumption, we postulate a simple relationship for how branched actin impacts contractile stress, given in the SI. 

The dynamics of rescue in this case are shown in Fig.\ \ref{fig:BAInStr}. In the early stages of rescue, we see a concentration of myosin and CDC-42 into an interior peak which grows over time. Once the concentration of CDC-42 at the peak exceeds the threshold $C_R=0.25$, branched actin is produced. This causes an immediate drop in contractility within the peak (see the active stress profile at intermediate times), which leads to a leveling out of the anterior profiles of CDC-42 and myosin (weaker flows in the anterior allow CDC-42 to diffuse over). Following this, branched actin builds up on the anterior. The boundary stops moving when the tension on the anterior balances the tension on the posterior, leaving an intermediate zone of contractility at around 60\% embryo length. 

A necessary feature of our simulations is the threshold dependence of branched actin production on CDC-42 activity.


\section*{Discussion}
%There is still a role for cytoplasmic depletion in pinning the boundary. Without cytoplasmic depletion, a boundary that is moving will always keep moving, since the fundamental balance in which pPARs outcompete the aPARs does not change unless we account for changes in the cytoplasmic depletion. What branched actin allows for is a change in how the flow speed depends on the myosin concentration in time. Initially, when there is no branched actin, flows are fast. But, when branched actin is created, the flows slow down and the boundary can no longer rapidly contract, allowing it to be stalled by cytoplasmic depletion. Thus, branched actin \emph{and} cytoplasmic depletion work together to stall the boundary at a point where the posterior PAR domain can no longer advance through the anterior PARs.

We note a distinction here between maintenance and establishment phase. In the latter, instabilities in the dynamics of rho combine with delayed negative feedback to yield pulsatile dynamics \citep{nishikawa2017controlling, michaux2018excitable, michaud2022versatile}. The connection between these pulsatile dynamics and the large-scale flows that establish polarity is still unknown.

\newpage 
\begin{center}
\includegraphics[width=\textwidth]{Cassandra/Fig1/Figure1.pdf}
\end{center}

\newpage 
Figure 1: Cell cycle progression controls a regime change from rho-dependent to CDC-42-dependent contractility. (a) Overview of dynamics during establishment and maintenance phase, including key events in the cell cycle. (b) Distributions of non-muscle-myosin II (NMY-2::GFP; left panels) GFP::PAR-6 (right panels) in wild-type embryos at/near the end of pseudocleavage (top panels) or nuclear envelope breakdown (bottom panels). Graphs show the normalized intensity of myosin foci and clusters (left), or PAR-6 clusters (right) over time relative to centration onset. (c) Kymogtaphs showing the correlation between pronuclear meeting and centration (left), psuedocleavage relaxtion (middle), and flow into the psuedocleavage (right) in wild type (left) and PRI-1 (RNAi) embryos. (d) Correlation between centration and psuedocleavage relaxation (left) and centration and myosin transitions (right) in wild-type (blue circles) and PRI-1 (RNAi) embryos (red squares). \red{First draft complete.}

\newpage
Figure 2: Maintenance phase biochemistry encodes a stable, attractive steady state with a unique boundary position. (a) CDK-1, (b) ect-2 ts, (c) Charlie's data.

\newpage 
\begin{center}
\includegraphics[width=\textwidth]{Cassandra/Fig3/Figure3.pdf}
\end{center}


\newpage
Figure 3: Models based a simple interaction of PAR proteins and myosin can stably maintain a polarity boundary, but cannot explain rescue. (a) Simplified model of PAR protein dynamics, and how it relates to the known system biochemistry \citep{lang2017proteins}. (b) Simulations without flow can maintain a stable boundary but cannot reproduce rescue in a realistic time. These panels show pseudo-kymographs of PAR-3 (green, left) and pPARs (red, right) in rescue (left, the boundary starts at 90\% embryo length) and normal maintenance (right, the boundary starts at 50\% embryo length) conditions. The bottom plot shows the unique steady state in the model without flow. (c) Simulations with flow lead to a vanishing anterior domain. Shown are pseudo-kymographs of the PAR-3, pPAR, and myosin concentrations over time, as well as the flow speeds. \red{First draft complete}

\newpage 
\begin{center}
\includegraphics[width=\textwidth]{Cassandra/Fig4/Figure4.pdf}
\end{center}

\newpage 
Figure 4: Branched actin acts in the anterior to suppress contractility in wild-type embryos. 

\newpage 
\begin{center}
\includegraphics[width=\textwidth]{Cassandra/Fig5/Figure5.pdf}
\end{center}

\newpage 
Figure 5: Branched actin acts suddenly to alleviate contractile tension in the anterior during maintenance phase rescue. (a) Still images and kymograph of myosin II (NMY-2::GFP) in ect-2 ts embryos (left) and ect-2 ts embryos with arx-2 (RNAi) (right). (b) Myosin intensity (i) and flow speeds (ii) in ect-2 (ts) embryos (blue lines) compared to wild-type. Data are shown at 30 second intervals using progressively darker lines, and wild-type data in the last minute of maintenance phase are shown in black. (c) Summary of rescue dynamics. Solid lines, governed by the left axis, show the maximum in myosin intensity (blue) and velocity (red), scaled by the value at the onset of maintenane phase ($t=0$). Dashed lnes, governed by the right axis, show the position of the peak in myosin (blue) and velocity (red) over time.  \red{First draft complete.}

\newpage 
\begin{center}
\includegraphics[width=\textwidth]{Cassandra/Fig6/Figure6.pdf}
\end{center}

\newpage
Figure 6: Models that incorporate branched actin successfully reproduce the dynamics of rescue. (a) Extension of the model presented in Fig.\ 3(a) to account for branched actin. Above a threshold of CDC-42, branched actin is produced and inhibits contractile stress. At its simplest level, these dynamics are an incoherent feed-forward loop (right). (b) Simulations with varying branched actin thresholds show varying degrees of contractility. Shown are kymographs of the PAR-3 domain (top) and flow speeds (bottom) over time. Each column shows a different threshold for branched actin, given in terms of the typical anterior CDC-42 concentration (0.4; see Fig.\ 3(b)). The steady state flow profiles for each threshold are also shown at bottom right. \red{First draft complete.}

\bibliographystyle{plainnat}

\bibliography{../../PolarizationBib}


\end{document}
