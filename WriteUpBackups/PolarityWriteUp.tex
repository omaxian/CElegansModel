\documentclass[11pt]{article}
\linespread{1.5} 
\usepackage{graphicx,epstopdf,subfigure,mathtools,mathrsfs, arydshln, amsmath, amssymb} 
\usepackage[font=small,labelfont=bf]{caption}
\usepackage{float}
\usepackage{authblk}
\usepackage[title]{appendix}
\PassOptionsToPackage{usenames,dvipsnames}{xcolor}
\usepackage[usenames,dvipsnames]{xcolor}
\usepackage[margin=1in]{geometry}
\usepackage[normalem]{ulem}

\usepackage{amsfonts}
\usepackage{hyperref}
\hypersetup{
    colorlinks=false,
    pdfborder={0 0 0},
}
\newcommand{\new}[1]{\color{blue}#1\normalcolor}
\newcommand{\red}[1]{\color{red}#1\normalcolor}
\newcommand{\delete}[1]{}
\newcommand{\change}[1]{\color{black}#1\normalcolor}
\newcommand{\rev}[1]{\color{black}#1\normalcolor}


% VECTOR AND MATRIX NOTATION
\newcommand{\V}[1]{\boldsymbol{#1}}                 % vector notation
\newcommand{\M}[1]{\boldsymbol{#1}}
\newcommand{\Lop}[1]{\boldsymbol {\mathcal{#1}}}
\global\long\def\Ac{A_\text{cyto}}
\global\long\def\Pc{P_\text{cyto}}
\newcommand{\CDC}[1]{#1_{\text{c}}}
\newcommand{\6}[1]{#1_{\text{6}}}
\newcommand{\3}[1]{#1_{\text{3}}}
\newcommand{\CHIN}[1]{#1_{\text{ch}}}
\global\long\def\kon{k^\text{on}}
\global\long\def\koff{k^\text{off}}
\global\long\def\kf{k^+}
\newcommand{\Tot}[1]{#1^\text{(Tot)}}
\global\long\def\Dt{\partial_t}
\global\long\def\Dthat{\partial_{\hat{t}}}
\global\long\def\Dx{\partial_x}
\global\long\def\Dxhat{\partial_{\hat{x}}}
\global\long\def\MChinC{P_\text{cyto}}
\global\long\def\MChin{P_1}
\global\long\def\PChin{P_n}
\global\long\def\MAC{A_\text{cyto}}
\global\long\def\MA{A_1}
\global\long\def\PA{A_n}
\global\long\def\CDCy{C_\text{cyto}}
\global\long\def\CD{C}
\global\long\def\kp{k^\text{p}}
\global\long\def\kdp{k^\text{dp}}
\global\long\def\kI{k^\text{I}}
\global\long\def\kE{k^\text{E}}
\newcommand{\A}[1]{#1_A}
\newcommand{\Chin}[1]{#1_P}
\newcommand{\C}[1]{#1_C}
\global\long\def\DhatA{\hat{D}_A}
\global\long\def\KhatonA{\hat{K}^\text{on}_A}
\global\long\def\KhatoffA{\hat{K}^\text{off}_A}
\global\long\def\KhatfA{\hat{K}^\text{+}_A}
\global\long\def\KhatpA{\hat{K}^\text{p}_A}
\global\long\def\KhatdpA{\hat{K}^\text{dp}_A}
\global\long\def\DhatP{\hat{D}_P}
\global\long\def\KhatonP{\hat{K}^\text{on}_P}
\global\long\def\KhatonM{\hat{K}^\text{on}_M}
\global\long\def\KhatoffP{\hat{K}^\text{off}_P}
\global\long\def\KhatoffM{\hat{K}^\text{off}_M}
\global\long\def\KhatfP{\hat{K}^\text{+}_P}
\global\long\def\KhatpP{\hat{K}^\text{p}_P}
\global\long\def\KhatdpP{\hat{K}^\text{dp}_P}
\global\long\def\DhatC{\hat{D}_C}
\global\long\def\KhatonC{\hat{K}^\text{on}_C}
\global\long\def\KhatoffC{\hat{K}^\text{off}_C}
\newcommand{\My}[1]{#1_M}
\newcommand{\R}[1]{#1_R}

\title{Modeling mechanochemical coupling in cell polarity establishment  \vspace{-0.5 cm}}
\author{Ondrej Maxian  \vspace{-0.75 cm}}

\begin{document}
\maketitle
This project is about understanding the design principles by which cells combine mechanics (the actomyosin network) and biochemistry to robustly polarize. 


\section{Dynamics of myosin in the embryo \label{sec:myosin}}
We begin by considering a model of myosin by itself, to determine if it can spontaneously polarize. Spontaneous polarization of myosin, if possible, would explain the maintenance-phase rescue experiments that we observe.

\subsection{Myosin as a self-patterning material} 
Let us begin by building a toy model for myosin dynamics. This section is a summary of the paper \cite{bois2011pattern}, which considers the same problem. The novelty in what we do will come later, when we couple myosin to branched actin and other proteins. 

We describe the dynamics of myosin $M(x,t)$ using the advection-diffusion equation
\begin{equation}
\label{eq:ADMy}
\Dt M + \Dx \left(v M\right) = D_M \Dx^2 M.
\end{equation}
The complication is that the myosin is advected through a velocity field of its own making. The velocity field comes from stress generated in the fluid, 
\begin{equation}
\label{eq:StrMy}
\sigma = \eta \Dx{v} + \sigma_a(M),
\end{equation}
which is a combination of viscous stress and active stress. As in \cite{bois2011pattern}, we ignore the elastic part of the stress, assuming the actomyosin cortex is purely viscous when in reality it is visco-elastic. The force balance equation in the fluid says that the force due to stress must be balanced by the drag force, 
\begin{equation}
\label{eq:FBMy}
\gamma v = \Dx \sigma,
\end{equation}
where $\gamma$ is the drag coefficient. Combining the force balance\ \eqref{eq:FBMy} with the stress expression\ \eqref{eq:StrMy} gives an auxillary equation for the velocity field
\begin{equation}
\gamma v = \eta \Dx^2 v + \Dx \sigma_a(M)
\end{equation}
which couples to the myosin equation\ \eqref{eq:ADMy} via the active stress.

The advection-diffusion equation\ \eqref{eq:ADMy} is mass-preserving, meaning that the uniform steady state is just given by $\displaystyle{M_0 = \frac{1}{L}\int_0^L M(x,0) \, dx}$. For the active stress, we let $\sigma_a=\sigma_0 \hat \sigma_a(M)$. The analysis of \cite{bois2011pattern} shows that (for periodic boundary conditions) the uniform steady state is unstable when  
\begin{equation}
\label{eq:StabBD}
\text{Pe}  \times \frac{M_0 \left(\partial_M \hat \sigma_a \left(M_0\right)\right)}{1+\left(2\pi \ell/L\right)^2} > 1,
\end{equation}
where $\ell=\sqrt{\eta/\gamma}$ is the characteristic lengthscale over which velocity decays (the diffusive lengthscale for velocity), $L$ is the system length, and the Peclet number
\begin{equation}
\text{Pe} = \frac{\sigma_0}{D_M \gamma}
\end{equation}
expresses the ratio of advective transport to diffusive transport. Qualitatively, the system has a uniform steady state and a second peaked steady state, where advective flux into the peaks matches the diffusive flux into the peaks. For this steady state to be stable, the advective transport must be sufficiently large relative to diffusive transport, so the Peclet number must be sufficiently large.

\iffalse
\subsubsection{Results}
We now consider a specific example where myosin generates active stress with $\sigma_a = \sigma_0 M$, and the initial condition is $M(x)=\cos{(2\pi x)}+1$ on $x \in [0,1]$. We set all parameters equal to 1, except for the diffusivity $D_M$. The stability result\ \eqref{eq:StabBD} says that, under these conditions, the uniform steady state is stable when $D >  0.0247$. 

\begin{figure}
\centering
\includegraphics[width=\textwidth]{MyosinSelfPattern.eps}
\caption{\label{fig:MyosinSP}Self-patterning of myosin. We consider a set of parameters for which the uniform steady state is stable only for $D > 0.0247$. Starting with the same initial condition $M(x,0)$ (blue), we show the steady states (red) for three different values of diffusivity (from left to right $D_M=0.025, D_M=0.02$, and $D_M=0.005$).}
\end{figure}

In Fig.\ \ref{fig:MyosinSP}, we therefore consider starting with the same initial condition and using three different values of the diffusivity. In the left panel, we use $D_M = 0.025$, for which the uniform steady state is stable, and the initial condition relaxes to that. In the middle panel, we show $D_M=0.02$, for which we see a relatively broad peak of myosin that takes up about half the cell. Dropping the diffusivity to $D_M=0.005$, there is less diffusive flux, and so the gradient of myosin must be larger for the diffusive flux to balance the advective flux. As a result we get a much sharper peak of myosin concentration in the rightmost panel. 
\fi


\begin{table}
\begin{small}
\centering
\begin{tabular}{|c|c|c|c|c|c|}\hline
Parameter & Description & Value & Units & Ref & Notes \\ \hline
$L$ & Domain length & 134.6 & $\mu$m &  \cite{goehring2011polarization} & radii $27 \times 15$ $\mu$m ellipse\\
$h$ & Cytoplasmic ``thickness'' & 9.5 & $\mu$m &  \cite{goehring2011polarization}  &  (area/circumference)\\ \hline
$\My{D}$ & Myosin diffusivity & 0.05 & $\mu$m$^2$/s &\cite{gross2019guiding} & \\
$\My{\kon}$ & Myosin attachment rate & 0.5 & $\mu$m/s & & Fit to get 30\% bound myosin\\
$\My{\koff} $ & Myosin detachment rate & 0.12 & 1/s & \cite{gross2019guiding}& \\
%$\Tot{M}$ & Maximum bound myosin density & 2600 & $\#/\mu$m &\cite{gross2019guiding} & $3.5 \times 10^5$ molecules/(134.6 $\mu$m) \\ \hline
$\Tot{M}$ & Maximum bound myosin density & -- & $\#/\mu$m & & Scales out of equations \\ \hline
$\eta$ & Cytoskeletal fluid viscosity & 0.1 & Pa$\cdot$s & &100 $\times$ water \\
$\gamma$ & Myosin drag coefficient & $10^{-3}$ & Pa$\cdot$s/$\mu$m$^2$ &  & $\ell=\sqrt{\eta/\gamma}=10 \, \mu$m \cite{saha2016determining}\\ 
$\sigma_0$ & Stress coefficient and form& 0.0042 & Pa & & Fit in Sec.\ \ref{sec:MyVelFit}\\
$\hat \sigma_a(\hat M)$ & Stress function of myosin& $\hat M$ & & & Fit in Sec.\ \ref{sec:MyVelFit}\\ \hline
\end{tabular}
\caption{\label{tab:paramsMy} Parameter values for myosin model. All of these parameters listed with a citation are lifted directly from the corresponding paper. Remaining parameters: the on rate $\My{\kon}$ is chosen to give 30\% bound myosin \cite[Fig.~S3]{gross2019guiding}. Later this rate will change in the presence of CDC-42. We make an assumption about the fluid viscosity $\eta$, which then gives us the drag coefficient $\gamma$ from $\ell=10$ $\mu$m \cite{gross2019guiding}. The remaining parameters are fit in Section\ \ref{sec:MyVelFit} from the wild-type data of \cite{sailer2015dynamic}. }
\end{small}
\end{table}


\subsection{Myosin pattern formation with turnover}
We now introduce a single species model of myosin with turnover, 
\begin{subequations}
\begin{gather}
\Dt M + \Dx \left(v M\right) = D_M \Dx^2 M +\My{\kon}M_\text{cyto} - \My{\koff} M \\
\label{eq:veleqndim}
\gamma v = \eta \Dx^2 v + \Dx \sigma_a(M)\\
M_\text{cyto} = \frac{1}{hL}\left(\Tot{M}L-\int_0^L M(x) \, dx\right)
\end{gather}
\end{subequations}
It will be useful to nondimensionalize this equation, using the scalings
\begin{equation}
\label{eq:NDD}
x = \hat{x} L \qquad t= \hat{t}/\My{\koff} \qquad M= \hat{M}\Tot{M} \qquad v = \hat{v} \frac{\sigma_0}{\sqrt{\eta \gamma}}
\end{equation}
The resulting equations are 
\begin{subequations}
\begin{gather}
\label{eq:MND}
\Dthat \hat{M} +\hat{\sigma}_0  \Dxhat \left(\hat{v} \hat{M} \right) =\hat{D}_M \Dxhat^2 \hat{M} +\hat{K}^\text{on}_M \left(1-\int_0^1  \hat{M}(x) \, dx\right) - \hat{M}\\
\label{eq:MVND}
\hat{v} = \hat{\ell}^2 \Dxhat^2 v +\hat{\ell} \Dxhat \hat{\sigma}_a(\hat{M})
 \end{gather}
\end{subequations}
and are controlled by the dimensionless parameters
\begin{equation}
\label{eq:NDparams}
\hat{\sigma}_0 = \left(\frac{\sigma_0/ \sqrt{\eta \gamma} }{L \My{\koff}}\right)   \qquad \hat{D}_M =\frac{D_M}{\My{\koff}  L^2} \qquad \hat{K}^\text{on}_M= \frac{\My{\kon}}{h \My{\koff}} \qquad \hat{\ell} = \frac{\sqrt{\eta/\gamma}}{L}.
\end{equation}
Recalling that $1/\My{\koff}$ is the residence time, these dimensionless parameters can be understood in the following way: 
\begin{enumerate}
\item $\hat{\sigma}_0$ is the fraction of the domain that active transport occurs on before a myosin molecule jumps off. To see this, note that residence time $\times$ the advective velocity $\sigma_0 / \sqrt{\eta \gamma}$ is the amount of motion, which is normalized by the domain length.
\item $\hat{D}_M$ is the maximum fraction of the domain a molecule diffuses before it unbinds (in the extreme case when the gradient in the domain is $1/L$, the diffusive velocity is $D_M/L$). 
\item $\hat{K}^\text{on}_M$ is the ratio of the binding rate to unbinding rate when all the molecules are cytoplasmic. The uniform steady state of the model is given by $\hat{M}_0= \hat{K}^\text{on}_M/\left(1+\hat{K}^\text{on}_M\right)$.
\item $\hat{\ell}$ is the ratio of the hydrodynamic lengthscale to the domain length.
\end{enumerate}
Prior to performing linear stability analysis, we need to first determine the function $\sigma_a$ and the other parameters. We do this in the next section by fitting experimental data.

\subsubsection{Parameter estimation}
Table\ \ref{tab:paramsMy} lists the parameters for the myosin model. According to \cite{goehring2011polarization}, the \emph{C.\ elegans} embryo has a roughly ellipsoidal shape, with half-axis lengths $27 \times 15 \times 15$ $\mu$m. As such, our model will be a $27 \times 15$ ellipse, which has perimeter $L=134.6$ $\mu$m. In our one-dimensional model, the cytoplasm has a ``thickness'' which is just the area of the ellipse $1272$ $\mu$m$^2$ divided by the perimeter $L$, which gives $h=9.5$ $\mu$m.

The next category of parameters relates to the myosin kinetics. The in-membrane diffusivity of myosin, as well as the detachment rate, have both been measured in \cite{gross2019guiding}. For the attachment rate, it was estimated in \cite[Fig.~S3m]{gross2019guiding} that roughly 30\% of myosin is bound to the cortex in wild-type embryos. Recalling that the uniform steady state is $\hat M_0=\hat{K}^\text{on}_M/\left(1+\hat{K}^\text{on}_M\right)$, this gives $\hat{K}^\text{on}_M=0.43$, or $\kon_M = 0.43h\koff_M=0.5$ $\mu$m/s. The last parameter, the total amount of myosin, scales out of the equations. This is fortunate for us because it is difficult to think about a total amount over a cross-section. 

For the fluid parameters, we assume that the viscosity of the cytoskeletal fluid on the cortex is 100 times water, which gives 0.1 Pa$\cdot$s. The ``hydrodynamic length scale'' of $\ell=\sqrt{\eta/\gamma}=10$ $\mu$m, measured in \cite{saha2016determining}, then gives the myosin drag coefficient $\gamma$. But more important than either of these is the stress as a function of myosin concentration. We fit this from the wild-type data of \cite{sailer2015dynamic} in the next section.

\subsubsection{Inferring flow profile from experiments \label{sec:MyVelFit}}
\begin{figure}
\centering
\includegraphics[width=\textwidth]{VelocityProfile.eps}
\includegraphics[width=\textwidth]{StressFromVelocityProfile.eps}
\caption{\label{fig:VelProf} Extracting the velocity profile and active stress from wild-type embryos. Top: the experimental data for myosin intensity (left) and velocity in $\mu$m/min (right). We show the raw data in black (which goes from anterior to posterior), the periodized version in blue, and a two-term (three terms if we include the constant) Fourier series representation in red. Bottom left: the recovered stress profile $\sigma_a(\hat x)$ in dimensional units. Bottom right: comparing the recovered stress to the myosin intensity, after normalizing by $\sigma_0=0.0011$ Pa. It is clear that $\hat \sigma_a = \hat M$ is a reasonable approximation.}
\end{figure}


Because we can measure the cortical velocity and myosin intensity, we can actually infer the function $\sigma_a(M)$ in dimensional units from the experimental data \cite{sailer2015dynamic}. We in particular isolate the myosin intensity and flow speed during ``late maintenance'' phase in wild type embroys \cite[Fig.~1B(bottom)]{sailer2015dynamic}, plotting the results in the top panels of Fig.\ \ref{fig:VelProf}. In the top left plot, we plot the myosin intensity, normalized so that the mean amount of bound myosin is 0.3, in accordance with wild-type measurements in \cite[Fig.~S3]{gross2019guiding}. 

In the top right plot, we show the velocity in $\mu$m/min. In both cases, the data are plotted on $\hat x \in [0.25,0.75]$, which corresponds to half of the embryo (one of the lines going from anterior to posterior end). We then periodically extend this data so that we fill the whole circumference $\hat x \in [0,1]$; these are the blue lines in Fig.\ \ref{fig:VelProf}. Finally, to remove the noise from our measurements (e.g., the strange dips in the myosin concentration at the anteior and posterior pole), we fit the periodized version with a two-term (+constant) Fourier representation, which gives the red lines in Fig.\ \ref{fig:VelProf}. 

To extract the stress profile from the smoothed velocity and myosin intensity, we consider a hybrid dimensional form of\ \eqref{eq:veleqndim}
\begin{equation*}
\gamma v -\frac{ \eta}{L^2} \Dxhat^2 v = \frac{1}{L} \Dxhat  \sigma_a(M). 
\end{equation*}
Let the Fourier series representation for $v(\hat x)= \sum_k \tilde v(k) \exp{\left(2 \pi i k \hat x \right)}$, and likewise for $\hat \sigma_a$. Then, in Fourier space, the solution for $\sigma_a$ is given by 
\begin{equation}
\label{eq:SigmaAF}
\sigma_a(k) = \frac{\gamma+ \eta/L^2 \left(2 \pi k\right)^2}{2 \pi i k/L} \tilde v(k). 
\end{equation}
The $k=0$ mode is undefined because $\sigma_a$ only appears differentiated; we therefore set it such that the real space stress has a minimum value of zero. 

\begin{figure}
\centering
\includegraphics[width=\textwidth]{VelocityProfileArx2.eps}
\includegraphics[width=\textwidth]{StressFromVelocityProfileArx2.eps}
\caption{\label{fig:VelProfArx2} Same plot as Fig.\ \ref{fig:VelProf}, but in \emph{arx-2} (RNAi) embryos. In the bottom right plot, we normalize by $\sigma_0=1.1 \times 10^{-3}$ Pa. This makes the stress (when shifted by an arbitrary constant) roughly the same as the myosin profile (also normalized so its maximum is 1).}
\end{figure}

We plug the parameters from Table\ \ref{tab:paramsMy} into\ \eqref{eq:SigmaAF} and show the resulting real space stress in the bottom left panel of Fig.\ \ref{fig:VelProf}. This is the dimensional stress $\sigma_a$. In the right panel of Fig.\ \ref{fig:VelProf}, we normalize and shift the stress so that it has the same mean and range as the myosin profile $\hat M$. Obtaining the same range allows us to read off the constant $\sigma_0=4.2 \times 10^{-3}$ Pa that controls the magnitude of the advective flows. In particular, the dimensionless parameter $\hat{\sigma}_0$ defined in\ \eqref{eq:NDparams} is seen to be equal to
\begin{equation}
\hat{\sigma}_0 = \left(\frac{\sigma_0/ \sqrt{\eta \gamma} }{L \My{\koff}}\right)  = 0.026.
\end{equation}
In additiom, the bottom right panel of Fig.\ \ref{fig:VelProf}, also shows that we can roughly set
\begin{equation}
\hat \sigma_a=\hat M
\end{equation}
as a good approximation to the stress. The function itself is ambiguous, since $\hat M=0.3$ defines two different values of the stress depending on the side of the domain, but $\hat \sigma_a=\hat M$ appears to be a good approximation.

We confirm this in Fig.\ \ref{fig:VelProfArx2}, where we repeat the velocity fitting procedure in \emph{arx-2} (RNAi) embryos, which lack branched actin and consequently have a simpler velocity profile. To compute the myosin profile, we assume that the experimentally-measured intensity can be converted to the dimensionless concentration $\hat M$ via the same factor (0.21) as wild-type embryos. Consequently, the myosin profile we obtain is in the top left of Fig.\ \ref{fig:VelProfArx2}. The velocity is shown in the top right panel, and we extract the stress profile in the bottom left in exactly the same way as in wild-type. Then, to compute normalize stress we divide out by $\hat{\sigma}_0=4.2 \times 10^{-3}$ Pa (obtained from wild-type). The normalized stress, when shifted by an arbitrary constant, lines up almost perfectly with the smoothed myosin profile, demonstrating that our rough approach from wild-type embryos extends to other embryos as well. Thus, this section gives us $\sigma_a = \left(4.2 \times 10^{-3}\right) \hat M$.


\subsubsection{Linear stability analysis \label{sec:StabMy}}
Now that all the parameters are known, we can perform linear stability analysis to see if the system could spontaneously polarize. The uniform steady state is $\hat{M}_0= \hat{K}^\text{on}_M/\left(1+\hat{K}^\text{on}_M\right)$. We consider a perturbation around that state $\hat M=\hat{M}_0+\delta \hat M$, where $\delta \hat M = \delta \hat M_0 e^{\lambda(k) \hat{t}+2 \pi i k \hat{x}}$. Plugging this into\ \eqref{eq:MVND}, we get the velocity \cite[Eq.~(11)]{bois2011pattern}
\begin{equation}
\hat v = \frac{2 \pi i k \hat{\ell} \hat{\sigma}'_a(\hat M_0)}{1 + \left(2 \pi k \hat \ell\right)^2} \hat \delta M. 
\end{equation}
Substituting this velocity into\ \eqref{eq:MND}, and considering only the first order terms, we get the following equation for the eigenvalues
\begin{equation}
\label{eq:DispRel}
\lambda(k) = \frac{4\pi^2 k^2 \hat{\ell} \hat{M}_0 \hat{\sigma}_0 \hat \sigma_a'(\hat{M}_0)}{1+4\pi^2 k^2 \hat{\ell}^2} - \hat{D}_M 4 \pi^2 k^2 -1
\end{equation}
Using the parameters we have obtained, we have the following values for the dimensionless groups
\begin{equation}
\hat{D}_M = 2.3 \times 10^{-5} \qquad \hat{M}_0 \approx 0.3 \qquad \hat \sigma_a'=1 \qquad \hat{\ell} \approx 0.07
\end{equation}
This gives the dispersion relation shown in Fig.\ \ref{fig:DispRelMy} for different values of $\hat{\sigma}_0$. We observe strong flow coupling required for instability; with $\hat{\sigma}_0=0.2$ (flow transports myosins around 20\% of the cell before they come off), we still do not see any instability. Considering that we already have seen wild-type embryos have $\sigma_0 \approx 0.004$, it is clear that myosin cannot self-polarize in the zygote.

Importantly, the large value of $\sigma_0$ needed for instability is a consequence of the $-1$ in the dispersion relation\ \eqref{eq:DispRel}, which comes from the unbinding kinetics. Thus, unbinding makes it \emph{harder} to destabilize the uniform steady state. Indeed, without the $-1$, the instability occurs at $\hat{\sigma}_0 \approx 10^{-3}$, which is pretty weak coupling to the flow. When we account for unbinding, diffusion becomes so small as to be irrelevant, as for the $k=1$ mode the coefficient in\ \eqref{eq:DispRel} is $\hat{D}_M 4 \pi^2 \approx 10^{-3}$. \textbf{Thus, the real balance here (to generate the instability) is not between advection and diffusion, but between advection and \emph{unbinding}}. Specifically, the advective flow must be strong enough to overcome the increase in unbinding that happens in areas enriched in myosin. 

\begin{figure}
\centering
\includegraphics[width=0.6\textwidth]{DispersionRelation.eps}
\caption{\label{fig:DispRelMy}Dispersion relation\ \eqref{eq:DispRel} for myosin for different values of $\hat{\sigma}_0$. Positive eigenvalues indicate instability of the steady state. Dotted black line at $\lambda=-1$ reflects the axis of instability \emph{without} unbinding kinetics.}
\end{figure}

\section{PAR-3 as the anchor for asymmetries \label{sec:Biochem}}
We are motivated first by the experimental observations that asymmetries in the PAR proteins are stable once set up, even in the absence of contractility. This experimental observation tells us that there is an intrinsic bistability in the biochemical circuit, which switches from a uniform state to a polarized state. In later sections, the switch will occur under the influence of actomyosin flows, while in this section the initial conditions will be the only way to switch the steady profiles.

Unlike in budding yeast cells \cite{mogilner2012cell}, there is no experimental evidence that \emph{C.\ elegans} cells can spontaneously polarize, which means that the system is truly bistable. Traditionally, it has been speculated that the bistability comes from mutual inhibition of the aPAR and pPAR proteins \cite{halatek2018self, trong2014parameter}. But translating this idea into equations becomes much harder than might be expected! Indeed, ODEs based on first-order mass action kinetics of aPAR-pPAR inhibition \emph{do not} yield bistable dynamics under any choice of parameters \cite{dawes20113}. Attempts to overcome this have used stoichiometric coefficients for the biochemical equations that guarantee bistability \cite{goehring2011polarization, gross2019guiding} or included actomyosin flows designed to transport the aPARs \cite{TH2008}. Both of these approaches are grounded more in intuition than in biological evidence, as there is no reason to doubt mass action kinetics, and recent experiments have shown that both aPARs and pPARs are transported by myosin \cite{illukkumbura2023design}. 

Recent experimental observations about PAR-3 provide a potential way out of this conundrum. Indeed, it was recently shown that PAR-3 asymmetries are stable even in the absence of its posterior inhibitor PAR-1, which suggest that the dynamics of PAR-3 \emph{by itself} are intrinsically bistable \cite{lang2023oligomerization}. Experimental evidence has shown that the bistability occurs via a mechanism in which membrane-bound PAR-3 recruits additional cytoplasmic monomers to the membrane. One goal of this section is to translate these observations into equations which demonstrate how PAR-3 can set up and maintain an asymmetry in the absence of posterior inhibition. We then incorporate posterior PAR proteins and show how their inclusion modifies the dynamics of PAR-3, potentially shifting the boundary between the two protein domains. 

\subsection{Basic equations and framework for PAR-3 \label{sec:Par3}}
We first formulate our model of PAR-3 dynamics, which is based loosely on that of Lang and Munro \cite{lang2022oligomerization}. The key property of PAR-3 that makes it different from other proteins is its ability to form \emph{oligomers} on the membrane. Unlike monomers, these oligomers do not diffuse in the membrane, and are not found in high concentrations in the cytoplasm. Based on these experimental observations, we will consider a model in which there are two species of PAR-3, 
\begin{enumerate}
\item Monomeric PAR-3, which can be found in cytoplasmic form $(\MAC)$ or membrane bound $(\MA)$ form.
\item Oligomerized PAR-3 $(\PA)$ which is only found on the membrane and can neither diffuse nor become unbound. These assumptions are approximations based on the experimental observations in \cite[Fig.~3K]{lang2023oligomerization}, which show that the dissociation rate constant for dimers in trimers is 5--10 times smaller than that for monomers, and also the experimental observation that PAR-3 only binds to the membrane in monomer form \cite{lang2023oligomerization}.
\end{enumerate}
Given these assumptions, the model equations in dimensional form are as follows
\begin{subequations}
\label{eq:P3model}
\begin{align}
\Dt \MA & = \A{D} \Dx^2 \MA + \left(\A{\kon}+\A{\kf}\A{f}^+\left(\MA,\PA\right)\right)  \MAC + 2\A{\kdp}\PA -2\A{\kp} \MA^2 - \A{\koff}\MA \\
\Dt \PA & =\A{\kp}\MA^2- \A{\kdp}\PA \\ \label{eq:Acyto}
A_\text{cyto} & = \frac{1}{h L} \left(\Tot{A} L - \int_0^L \left(A_1(x)+2A_n(x)\right) \, dx \right)
\end{align}
\end{subequations}
A complete list of parameters with units and values is given in Table\ \ref{tab:params}, but it will be helpful to point out the important ones in our model. First, the feedback strength $k_A^+$, which has units of length$^2$/time, gives the rate at which cytoplasmic PAR-3 is recruited to the membrane. It is multiplied by the flux function $f_A^+$, which gives the strength of recruitment (in units of inverse length; this could be as simple as $f_A^+(x,y)=x+2y$). The overall on rate is proportional to the cytoplasmic concentration, which is defined in\ \eqref{eq:Acyto}. There $\Tot{A}$ expresses the density of bound PAR-3 when all molecules are bound to the membrane (units 1/length). Subtracting the amount of bound PAR-3 and dividing by the membrane area gives
the cytoplasmic concentration in units of 1/area.

\subsection{Dimensionless form}
A sensible timescale for the system is the time a given PAR-3 molecule spends on the membrane. Because about 80\% of the bound PAR-3 molecules are in oligomer form, and since the depolymerization reaction is much slower than the unbinding reaction, we nondimensionalize time by $1/\A{\kdp}$. This gives the dimensionless (hatted) variables defined by
\begin{equation*}
x = \hat{x} L \qquad t= \hat{t}/\A{\kdp} \qquad A= \hat{A}\Tot{A}.
\end{equation*}
Substituting into\ \eqref{eq:P3model} gives the rewritten dynamics
\begin{subequations}
\label{eq:PAR3gen}
\begin{align}
\nonumber
\Dthat \hat{\MA} & = \DhatA \Dxhat^2 \MA +\KhatonA \left(1+\KhatfA \hat F_A^+(\hat \MA, \hat \PA) \right)\left(1 - \int_0^1 \hat A(x) \, d\hat{x} \right) \\ 
\label{eq:Amono} &+ 2\KhatdpA \hat{\PA}-2\KhatpA \hat{\MA}^2 - \KhatoffA \hat{\MA} \\
\label{eq:Apoly}
\Dthat \hat{\PA} & =\KhatpA \hat{\MA}^2- \KhatdpA \hat{\PA} \\ 
\label{eq:paramsND}
\DhatA &=\frac{\A{D}}{L^2 \A{\kdp}}, \quad \KhatonA=\frac{\A{\kon}}{\A{\kdp} h }, \quad \KhatfA = \frac{\A{\kf}\Tot{A}}{\A{\kon}}, \quad  \KhatoffA = \frac{\A{\koff}}{\A{\kdp}}, \\ \nonumber \KhatpA &= \frac{\A{\kp} \Tot{A}}{\A{\kdp}}, \quad \KhatdpA=1, \quad \hat F_A^+(\hat \MA,\hat \PA) =\frac{\A{f}^+(\MA,\PA)}{\Tot{A}}
\end{align}
\end{subequations}
and we have used the convenient shorthand $A=\MA + 2 \PA$ and likewise for the hatted variables. 

\subsection{Parameters \label{sec:paramsP3}}
Recent experimental measurements \cite{lang2023oligomerization} give accurate measurements for three of the parameters: the diffusion coefficient $D_A$, the detachment rate of monomers $\koff_A$, and the depolymerization rate $\kdp_A$. The values of these parameters are summarized in Table\ \ref{tab:params}, and determine the dimensionless parameters $\DhatA$ and $\KhatoffA$. We determine $\KhatonA$, $\KhatpA$, $\KhatfA$, and the form of the feedback function via a systematic fitting procedure as detailed next.

\subsubsection{Fitting the polymerization rate}
The first parameter we need to fit the polymerization rate is the percentage of PAR-3 bound to the membrane in the uniform state. The uniform state can be realized by considering mutants which lack a functional sperm cue and thus do not polarize \cite[Fig.~S1]{lang2023oligomerization}. These mutants show a peak PAR-3 concentration in late interphase; late maintenance phase then gives a bound concentration that is roughly 50\% of this peak. We will assume that almost all of the PAR-3 is bound in late interphase, so that the uniform state is at $\hat A_u:=\hat{\MA}+2\hat{\PA} \approx 0.5$. The observation that these embryos do not polarize implies that the uniform state is stable, and the estimate for the percentage of bound protein is similar to that obtained for PAR-2 at the end of maintenance phase in \cite[Fig.~S3]{gross2019guiding}. 

When there is no posterior inhibitor, the concentration of bound PAR-3 during late maintenance phase in the anterior is roughly $\hat A_a=1.2u \approx 0.6$ (this comes from comparing flourescence in PAR-1 mutant and spd-5 mutant embryos shown in Figs.\ 2 and S1 of \cite{lang2023oligomerization}). In PAR-1 mutants, the concentration in the posterior is then roughly 10\% of the anterior, or $\hat A_p=0.06$. 

These measurements allow us firstly to determine the relative polymerization rate $\KhatpA$. The distribution of oligomer sizes in PAR-1 mutant embryos was measured in \cite{lang2023oligomerization} on both the anterior and posterior side. There it was shown that the distribution is roughly exponential, so that $a_n$, the number of size $n$ is $a_n=\alpha^{n-1} a_1$. This implies that the fraction in monomer form is $\left(1-\alpha\right)^2$. Using the measurements in \cite{lang2023oligomerization}, we have $\alpha=0.42$ on the posterior side (30\% in monomer form), and $\alpha=0.73$ on the anterior side (10\% in monomer form) by solving \eqref{eq:Apoly} at steady state to obtain the relationship
\begin{equation}
\frac{\hat A_1}{\hat A} = \frac{1}{4 \hat A \KhatpA}\left(-1+\sqrt{1+8 \hat A \KhatpA}\right).
\end{equation}
We then insist that $\hat A=0.6 \rightarrow \hat A_1=0.06$ and solve for $\KhatpA$, which gives $\KhatpA=75$. The corresponding value on the posterior side is $\hat A = 0.06$ and $\hat A_1=0.017$, which is roughly 28\% of the protein in monomer form, matching the experimental observation.

\subsubsection{The form of the feedback strength -- linear feedback models}
Before we get into fitting the feedback parameters, it is important to consider the nature of the feedback model. Our model is based strongly on that of Lang and Munro \cite{lang2022oligomerization}, who used the linear feedback model $$f_A^+(\MA,\PA)=\MA+2\PA=A.$$ 
To analyze the characteristics of this model, we consider two representative examples in Fig.\ \ref{fig:P3Linear}, where we look at the attachment and detachment fluxes as a function of $\hat A$. The detachment flux is simply $\KhatoffA \hat \MA$, and is therefore set by the red line. The attachment flux varies depending on the model considered. If we consider a uniform state, then the cytoplasmic concentration is simply $1-\hat A$, and there is a single uniform steady state (intersection of dotted blue and red lines). In polarized states, the concentration is not necessarily uniform, and so we analyze the stability of the steady state by taking the cytoplasmic concentration as constant. This results in the darker blue line in Fig.\ \ref{fig:P3Linear}. There we see that the linear feedback model admits only two possibilities: a stable uniform state (when feedback is small relative to the on-rate), and an unstable uniform state which leads to spontaneous polarization (when the feedback is larger). This contradicts our experimental observation of \emph{bistability}.

\begin{figure}
\centering
\includegraphics[width=\textwidth]{FluxAnalysisLinFeed.eps}
\caption{\label{fig:P3Linear}Flux plane analysis for linear feedback in the stable (left) and unstable (right) case. The stability analysis is determined by how the attachment rate (solid blue line, with constant cytoplasmic concentration) compares to the detachment rate (red) near the steady state.}
\end{figure}

For the uniform steady state to be stable, the attachment rate at constant cytoplasmic concentration has to be smaller than the detachment rate, as shown in the left panel of Fig.\ \ref{fig:P3Linear}. At the same time, for bistability, the attachment flux at constant cytoplasmic concentration has to intersect the detachment curve three times (two stable fixed points and one unstable fixed point in between). A simple way to accomplish this is by setting
\begin{equation}
f_A^+(\MA,\PA) =\min\left(\MA+2\PA,A_\text{sat}\right) \rightarrow F_A^+(\hat \MA, \hat \PA) = \min\left(\hat \MA+2\hat \PA, \hat A_\text{sat}\right). 
\end{equation}
The uniform steady state is stable if $\hat A_\text{sat} < u = 0.5$, which provides one constraint on the saturation. The second constraint comes from bistability; the system is only locally bistable at fixed cytoplasmic concentration when $\hat A_\text{sat}$ is close to the uniform state (if $\hat A_\text{sat}$ is too small the feedback becomes essentially constant and cannot generate bistability). Based on these considerations, we set $\hat A_\text{sat}=0.4$. 


\begin{figure}
\centering
\includegraphics[width=0.6\textwidth]{FluxAnalysisLinFeedCap.eps}
\caption{\label{fig:P3Cap}Flux plane analysis for capped linear feedback in the bistable case ($\KhatfA=12$ and $\kon_A=0.5$ $\mu$m/s). The stability analysis is determined by how the attachment rate (solid blue line, with constant cytoplasmic concentration) compares to the detachment rate (red) near the steady state. }
\end{figure}

Once $\hat A_\text{sat}=0.4$ is set, there are two unknown parameters, the on rate $\kon_A$ and the strength of the feedback $\KhatfA$. In Fig.\ \ref{fig:P3Cap}, we set these two parameters so that, at the cytoplasmic concentration of the uniform state, the system has a second steady state at roughly 10\% of the uniform state. The bistable solution exists for a narrow range of cytoplasmic concentration -- when the cytoplasmic concentration is too small or too large, the attachment flux only crosses the detachment flux once, which means that there is no stable state for these cytoplasmic concentrations. Suppose we have the cytoplasmic concentration $A_c$, and let $\hat A_1$ and $\hat A_2$ be the two stable states at this concentration. Then in the absence of diffusion, if the equation 
\begin{equation}
\label{eq:NoDiffEqn}
(1-y)\hat{A}_1\left(A_c\right)+y \hat{A}_2\left(A_c\right)=1-A_c
\end{equation}
has a solution, then $y$ defines the boundary position. Otherwise, there will be a uniform state.

This completes our parameter selection for the PAR-3 model. The parameters we use going forward are summarized in Table\ \ref{tab:paramsP3}.


\begin{table}
\begin{small}
\centering
\begin{tabular}{|c|c|c|c|c|c|}\hline
Parameter & Description & Value & Units & Ref & Notes \\ \hline
$\A{D} $ & Monomeric PAR-3 diffusivity & 0.1 & $\mu$m$^2$/s & \cite{lang2023oligomerization} & \\
$\A{\kon}$ & Monomeric PAR-3 attachment rate & 0.5 & $\mu$m/s & & Fit for uniform state $\hat A=0.5$ \\
$\A{\koff}$ & Monomeric PAR-3 detachment rate &  3& 1/s & \cite{lang2023oligomerization} & (Fig.\ 3K)\\
$\A{\kdp}$ & PAR-3 depolymerization rate & 0.08 & 1/s & \cite{lang2023oligomerization} & (Fig.\ 4E) \\
$\KhatpA$ & PAR-3 polymerization rate & 75 & & & Fit for correct \% monomers \cite{lang2023oligomerization}  \\
$\KhatfA$ & PAR-3 self recruitment rate &12 & & & Fit for bistability\\
$F_A^+$ & PAR-3 feedback function &$\text{min}\left(\hat A,0.4\right)$ &  &\cite{lang2022oligomerization} &  Modified for stable uniform state\\
$\Tot{A}$ & Maximum bound PAR-3 density & -- & $\#/\mu$m & & Contained in other unknowns \\ \hline
\end{tabular}
\caption{\label{tab:paramsP3}Additional parameter values for the PAR-3 model. }
\end{small}
\end{table}

\subsection{Results for bistable PAR-3 model}

\subsubsection{Steady states}
\begin{figure}
\centering
\includegraphics[width=\textwidth]{PAR3Boundary.eps}
\caption{\label{fig:P3FBBd}Simulating the PAR-3 feedback model with the parameters in Table\ \ref{tab:paramsP3}. The initial conditions are shown in dotted blue, and the distribution at $\hat t = 500$ ($\hat t =1$ corresponds to 12.5 seconds of real time) is in red. In the top row, we make a continuous perturbation from the uniform state, finding that large enough perturbations induce bistability. In the bottom row, we start with a peaked initial profile of large and small size, finding that only larger (than 0.5) initial profiles can lead to bistability.}
\end{figure}

We now look at the steady states of the PAR-3 model with the parameters in Table\ \ref{tab:paramsP3}. Figure\ \ref{fig:P3FBBd} shows what happens when we run from $\hat t = 0$ to $\hat t = 500$, which achieves the steady state. In the top left plot, we see that the uniform steady state is stable, as expected from the stability diagram. But when the perturbation to the uniform state is too large, or when we introduce an asymmetry into the system by depleting PAR-3 in part of the domain, we see bistable dynamics where the small part gravitates to one steady state, while the larger end goes to another. In the bistable region, we observe a posterior concentration which is always roughly 10\% of the anterior concentration, as desired (the exact number at steady state is $0.046/0.49 =9.3$\%. 

In the case when we deplete PAR-3 in part of the domain, the bistable behavior only happens when the initial domain size is sufficiently large. Figure\ \ref{fig:P3FBBd} demonstrates this in the bottom row, where we consider initial domains of PAR-3 enrichment of 0.9 (bottom left) and 0.5 (bottom right). We find that when the initial PAR-3 domain of enrichment is larger than 50\% of the domain, the system tends to the bistable state, with about 95\% of the domain enriched in PAR-3 and 5\% at the lower state. When the initial domain of PAR-3 enrichment is too low, however, we find that the flux into the depleted regions is too large, and those regions tend to surpass the smaller uniform steady state and be attracted to the larger one. The higher flux happens because of a larger cytoplasmic concentration (which could result from the initial condition, or from unbinding from the enriched domain if we try to deplete the cytoplasm initially). In any case, the conclusion of Fig.\ \ref{fig:P3FBBd} is that there is a uniform steady state, which is the attractor when most of the PAR-3 is initially in the cytoplasm, and a bistable state, which is an attractor when the cytoplasmic concentration is around 0.5 (50\% of protein in the cytoplasm). 

\subsubsection{Position of boundary and approach to steady state}

\begin{figure}
\centering
\includegraphics[width=\textwidth]{PAR3DomainSizes.eps}
\caption{\label{fig:WWoDiff}Size of PAR-3 domain over time without (left) and with (right) diffusion of monomeric PAR-3. Without diffusion, any boundary position with 70\% or more enriched PAR-3 is stable, because there exists a solution to\ \eqref{eq:NoDiffEqn} where the on rate balances the off rate in both the enriched and depleted regions. When we introduce diffusion, there is an additional constraint in the boundary layer which specifies a unique boundary position.}
\end{figure}

Let us now try to understand the position of the boundary. In Fig.\ \ref{fig:WWoDiff}, we plot the size of the enriched PAR-3 domain over time for various initial boundary positions. We start without diffusion, observing that, for sizes of the enriched PAR-3 domain 0.5 or larger, the cytoplasmic concentration is sufficiently low for a bistable solution to exist (left panel of Fig.\ \ref{fig:WWoDiff}). When the PAR-3 domain is initially too small, the cytoplasmic concentration at steady state would be too high for bistability, and so the uniform state is the only stable one. If the bistable solution exists, then the boundary position does not move in time; any domain size 0.5 or larger appears to be stable.

This is not the case when we add diffusion into the model, which provides an additional constraint. In this case, the right panel of Fig.\ \ref{fig:WWoDiff} shows that there is a \emph{unique} boundary position that the system tends to, when the diffusive flux in the \emph{boundary layer} balances the net unbinding and binding fluxes. Indeed, when we turn on diffusion, there is diffusion of monomers away from the enrichment zone, and consequently local depletion of monomers at the edge of the enrichment zone (oligomers are not depleted, so the on flux is basically unchanged). There is then an imbalance of flux where the flux from binding is larger than the unbinding flux at the edge of the enrichment zone. If the on rate dominates diffusion (this is the case when the cytoplasm is enriched), then the boundary will tend to expand. On the other hand, if diffusive fluxes are sufficiently large (cytoplasmic depletion or larger zone of enrichment), then the boundary will contract if diffusive flux outwards overcomes the increased local binding. Figure\ \ref{fig:ExpContrBd} shows examples of this.

\begin{figure}
\centering
\includegraphics[width=\textwidth]{ExpandContractBd.eps}
\caption{\label{fig:ExpContrBd}Local depletion by diffusion and consequent expansion/contraction of boundary. These figures show the positive fluxes due to binding (blue) and the negative fluxes due to diffusion and unbinding (red). When the boundary is narrow (left figure), the binding rate is higher due to enriched cytoplasm, and the boundary expands. On the other hand, when the boundary is wider (right figure), the binding rate is locally lower than the off rate + diffusion (note diffusion makes the difference in this case), and the boundary contracts. }
\end{figure}

Another observation from Fig.\ \ref{fig:ExpContrBd} is the \emph{rate} at which the boundary shifts to the steady state. Because the rate of expansion/contraction is controlled by diffusion of monomers, and because the dimensionless diffusivity $\DhatA \approx 7 \times 10^{-5}$,  the movement of the boundary is quite slow. Indeed, most embryos are in maintenance phase for about 10 minutes, which corresponds to $\hat t = 48$. During this time, the movement of the boundary is at most 5\% of the perimeter, or about 10\% of the embryo length. \red{Similar arguments to this were made in \cite{lang2022oligomerization} to justify the possibility that the uniform steady state is unstable, and that the polarized PAR-3 state (in which PAR-3 is stably enriched on one half of the embryo) \cite{lang2023oligomerization} might be a transient state that transitions to a large peak if we were to wait long enough. Here we propose that the polarized state of PAR-3 in the absence of PAR-1 is a result of the initial sperm cue, and that, if given long enough, the position of the boundary would relax towards the posterior side. }

\subsection{PAR-3 / PAR-2 bistable model \label{sec:P2P3}}
Let's now add posterior PARs (a single variable $P$ which represents the PAR-1/PAR-2 conglomerate) to the model of PAR-3, so that the equations are 
\begin{subequations}
\label{eq:P32}
\begin{align}
\Dt A_1 &= D_A \Dx^2 A_1 +\left( \kon_A +\A{\kf}\A{f}^+\left(\MA,\PA\right)\right) A_\text{cyto} - \koff_A \MA -2\kp_A \MA^2+2\kdp_A \PA,\\
\Dt \PA &= \kp_A \MA^2 - \kdp_A \PA-r_\text{PA} \PA P\\
\Dt P &= D_P \Dx^2 P + \kon_P P_\text{cyto} - \koff_P P -r_\text{AP} P\left(\MA+2\PA\right),%\\
%\Dt P_2 &= \kp_P P_1^2 - \kdp_P P_2-r_\text{AP} P_2\left(\MA+2\PA\right).
\end{align}
\end{subequations}
Here we have assumed that PAR-3 inhibits PAR-2 (by activating PKC-3, which is not included here for simplicity) \cite{lang2017proteins}. It is possible here to model PAR-2 in terms of both dimers and monomers \cite{bland2023optimized}; however, because PKC-3 acts on both monomers and dimers, the two are really only separated by kinetics of dimerization. 

\begin{table}
\begin{small}
\centering
\begin{tabular}{|c|c|c|c|c|c|}\hline
Parameter & Description & Value & Units & Ref & Notes \\ \hline
$D_P$ & PAR-2 diffusivity & 0.15 & $\mu$m$^2$/s & \cite{goehring2011polarization}&  \\ 
$\kon_P$ & PAR-2 attachment rate & 0.6 & $\mu$m/s & & 90\% bound PAR-2 no inhibition\ \\ 
$\koff_P$ & PAR-2 detachment rate & $7.3 \times 10^{-3}$ & 1/s & \cite{goehring2011polarization}&  \\  
$\hat{R}_\text{AP}$ & Rate of $P$ inhibition by $A$ & $100$ & & & Fit small $P$ in anterior\\
$\hat{R}_\text{PA}$ & Rate of $A$ inhibition by $P$ & 0.5  &  &  & Fit boundary position\\
$\Tot{P}$ & Maximum bound PAR-2 density & -- & $\#/\mu$m& &  Included in $\hat{R}_\text{PA}$ \\ \hline
\end{tabular}
\caption{\label{tab:paramsP2}Additional parameter values for the PAR-3 model when PAR-2  is also included.}
\end{small}
\end{table}

Repeating our non-dimensionalization from\ \eqref{eq:PAR3gen}, the dimensionless form of the equations\ \eqref{eq:P32} is
\begin{subequations}
\label{eq:PAR2PAR3}
\begin{align}
\nonumber
\Dthat \hat{\MA} & = \DhatA \Dxhat^2 \MA +\KhatonA \left(1+\KhatfA \A{F}^+\left(\hat{\MA},2\hat{\PA}\right)\right)\left(1 - \int_0^1 \hat{A}(x) \, d\hat{x} \right) \\ 
&+ 2\KhatdpA \hat{\PA}-2\KhatpA \hat{\MA}^2 - \KhatoffA \hat{\MA} \\ 
\Dthat \hat{\PA} & =\KhatpA \hat{\MA}^2- \KhatdpA \left(1+\hat{R}_\text{PA}\hat{P}\right) \hat{\PA} \\ 
\Dthat \hat{P} & =\DhatP \Dxhat^2 \hat{P} +\KhatonP \left(1 - \int_0^1 \hat{P}(\hat x) \, d\hat{x} \right)  - \KhatoffP \left(1+\hat{R}_\text{AP}\hat{A}\right)\hat{P} \\ 
\label{eq:paramsNDFB}
\hat{R}_\text{PA}&=\frac{r_\text{PA} \Tot{P}}{\kdp_A},\quad \hat{R}_\text{AP}=\frac{r_\text{AP} \Tot{A}}{\koff_P},\quad \DhatP =\frac{D_P}{L^2 \A{\kdp}}, \quad \KhatonP=\frac{{\kon_P}}{\A{\kdp} h },\quad \KhatoffP = \frac{\koff_P}{\kdp_A}
\end{align}
\end{subequations}
The additional dimensionless parameters are in\ \eqref{eq:paramsNDFB}. All other dimensionless parameters have already been defined and determined in Section\ \ref{sec:paramsP3}.

\subsubsection{Parameters \label{sec:param23}}
There are therefore five parameters that are unknown in the model\ \eqref{eq:PAR2PAR3}. The diffusion and dissociation rates of PAR-2 have been measured in \cite{goehring2011polarization}, and are reported in Table\ \ref{tab:paramsP2}. The mutual inhibition parameters and on rate $\kon_P$ can be fit with the following three experimental observations:
\begin{enumerate}
\item In embryos without myosin flows, the level of PAR-2 at the anterior is no more than 5\% of the posterior level \cite[Fig.~2c]{gross2019guiding}. This sets $\hat R_\text{AP} \gg 1$. We use $\hat R_\text{AP}=100$ for strong inhibition.
\item In embryos without myosin flows, roughly 25--30\% of the available PAR-2 is bound at steady state \cite[Fig.~S3]{gross2019guiding}. Because the PAR-2 domain is only 25--30\% of the embryo, the on rate in the absence of inhibition must be such that almost all of the cytoplasmic protein is bound. This sets $\kon_P=0.6$ $\mu$m/s (90\% bound no inhibition).
\item In embryos without myosin flows, the steady state boundary is at 70--75\% PAR-3 enrichment \cite[Fig.~5B]{zonies2010symmetry}. This sets $\hat{R}_\text{PA}=0.5$.
\end{enumerate}


\begin{figure}
\centering
\includegraphics[width=\textwidth]{PAR3PAR2FixedBd.eps}
\caption{\label{fig:P32FixBd}Steady states ($\hat t=500$) of the PAR-2/PAR-3 model\ \eqref{eq:PAR2PAR3} with various values of $\hat R_\text{PA}$ ($P$ inhibits $A$). The dotted blue and red lines shows the initial distribution of PAR-3 and PAR-2 (respectively) in each case, and the solid lines show the steady state at $\hat t = 500$. The title indicates the strength that $P$ inhibits $A$, which is $\hat R_\text{PA}$, and also the resulting size of the PAR-3 domain.}
\end{figure}

We try to tune the inhibition parameter in Fig.\ \ref{fig:P32FixBd} to achieve the correct boundary position at steady state. When there is no inhibition (left plot), PAR-3 develops its bistable state naturally, and we can tune the distribution of PAR-2 by changing $\hat R_\text{AP}$ and $\kon_P$ (not shown). Introducing nonzero $\hat R_\text{PA}$ then causes the boundary to shift, as depletion of PAR-2 in the interior leads to more cytoplasmic PAR-2, which increases the on rate in the inhibited zone. The attaching PAR-2 then diffuses, out-competing the PAR-3 and driving it to a lower steady state. This shifts the boundary, as previously observed in \cite{goehring2011polarization}. 

\subsubsection{Steady state}
Because of the experimental observation that embryos depleted of myosin have about 75\% PAR-3 enrichment at the end of maintenance \cite[Fig.~5B]{zonies2010symmetry}, we set $\hat R_\text{PA}=0.5$, which is the bottom left plot in Fig.\ \ref{fig:P32FixBd}. In this steady state, we have about 5\% (0.03/0.54) PAR-3 on the posterior relative to the anterior, and 7\% (0.1/1.4) PAR-2 on the anterior relative to the posterior. These roughly correspond to experimental observations, although the changes in the PAR-3 concentration (from 10\% to 5\%) are not as drastic as those observed experimentally \cite[Fig.~4C]{sailer2015dynamic}.

\subsubsection{Approach to steady state}

\begin{figure}
\centering
\includegraphics[width=0.6\textwidth]{PAR3SizePAR2Comp.eps}
\caption{\label{fig:P32FixBdEq}Size of the PAR-3 enrichment zone over time. Simulations use parameters in Table\ \ref{tab:paramsP3} and\ \ref{tab:paramsP2} and start with $\hat A=1$ and $\hat P=0$ in an enrichment zone (as in the bottom row of Fig.\ \ref{fig:P32FixBd}). We then watch the size of the domain evolve over time. If there is sufficient cytoplasmic depletion of PAR-3, then mutual inhibition by PAR-2 shifts the intrinsic boundary.}
\end{figure}

We now demonstrate that the steady state boundary position is set uniquely by the parameters. When $\hat R_\text{AP}=100$ and $\hat R_\text{PA}=0.5$, Fig.\ \ref{fig:P32FixBd} shows that the domain of PAR-3 enrichment at steady state is roughly 75\%. We confirm that this is a unique steady state in Fig.\ \ref{fig:P32FixBdEq} by changing length of the region initially enriched in PAR-3. We find that, as long as the domain size is larger than 0.2, we see a unique steady state where PAR-3 and PAR-2 mutually exclude each other. By contrast, when the initial domain of enrichment is too small, there is sufficient cytoplasmic PAR-3 for the uniform steady state to be the only stable one. These dynamics are similar to the case of PAR-3 alone (Fig.\ \ref{fig:WWoDiff}), except that inhibition by PAR-2 increases the range of cytoplasmic concentrations for which bistability exists. 

\red{We note also that, as in the intrinsic PAR-3 dynamics, the approach to steady state is quite slow, as it is contolled by the diffusion of PAR-3 monomers and PAR-2. Indeed, Fig.\ \ref{fig:P32FixBdEq} shows that, in 10 minutes of real time ($\hat t = 48$), the enriched PAR-3 domain changes by at most 5--10\%. }

\subsubsection{With the experimental myosin profile}
\begin{figure}
\centering
\includegraphics[width=\textwidth]{PAR23ExpFlow.eps}
\caption{\label{fig:P32ExpFlow}Augmenting the biochemistry model with the steady state maintenance phase flow in \emph{arx-2} (RNAi) embryos (see Fig.\ \ref{fig:VelProfArx2}). The left panel compares the steady states with PAR competition vs.\ the steady states with the experimental flow field. There is an enrichment of aPARs and depletion of pPARs as the boundary contracts inward. The speed of contraction for different initial conditions is shown at right. }
\end{figure}


To illustrate what happens when we add flow, we now add to the simulation the experimental flow profile in embryos depleted of branched actin, depicted in Fig.\ \ref{fig:VelProfArx2}. The model equations are identical to\ \eqref{eq:Everything}, except that the flow profile $\hat v$ is imposed from the experimental measurement in Fig.\ \ref{fig:VelProfArx2} and not computed internally. Figure\ \ref{fig:P32ExpFlow} shows the resulting steady state and the approach to it from different initial zones of PAR-2/3 enrichment. We see two major results: first, the boundary has shifted from 75\% PAR-3 enrichment to roughly 45\%, as is typical in the \emph{arx}-2 embryos. Second, the approach to the steady state is faster. Even though the boundary has to shift farther, the steady state is reached in almost all cases by $\hat t = 250$, as now flow \emph{and} diffusion control how long it takes to get there. Indeed, for the boundary to move 30\% of the domain length on each side, it needs to move 40 $\mu$m. Since the maximum flow speed is 1 $\mu$m/min, we expect 40 minutes, or 2400 seconds, or $\hat t \approx 200$ to be the time it takes to shift the boundary to steady state. This is indeed what we see in the right panel of Fig.\ \ref{fig:P32ExpFlow}. This indicates that flow dominates the approach to steady state.

\iffalse
\subsection{Model of PAR-2 and PAR-3 with myosin  \label{sec:P2P3My}}
Let's now add myosin to the model of PAR-2 and PAR-3 that we formulated in\ \eqref{eq:PAR2PAR3}. \red{We can't have $A$ promoting $M$ because there's nothing to stop the flow. Just do posterior inhibition and move on.} Myosin enters here as an advective flow for each of the proteins, then has its own constitutive equation. The full dimensionless model is given by
\begin{subequations}
\label{eq:PAR2PAR3My}
\begin{align}
\nonumber
\Dthat \hat{A}_1 +\hat{\sigma}_0  \Dxhat \left(\hat{v} \hat{\MA} \right)  &=\DhatA \Dxhat^2 \hat{A}_1 +\KhatonA\left(1+\hat{K}_A^+ \hat F_A^+\left(\hat{\MA},\hat{\PA}\right)\right)\left(1 - \int_0^1 \hat{A}(x) \, d\hat{x} \right) \\  
 &-\KhatoffA \hat{A}_1 +2\KhatdpA \hat{\PA}-2 \KhatpA \hat{A}_1^2 \\ 
\Dthat \hat{\PA} +\hat{\sigma}_0  \Dxhat \left(\hat{v} \hat{\PA} \right)  &=\KhatpA \hat{\MA}^2- \KhatdpA \left(1+\hat{R}_\text{PA}\hat{P}\right)\hat{\PA} \\ 
\Dthat \hat{P} +\hat{\sigma}_0  \Dxhat \left(\hat{v} \hat{P} \right) & =\DhatP \Dxhat^2 \hat{P}+\KhatonP \left(1 - \int_0^1 \hat{P}(\hat x) \, d\hat{x} \right)  - \KhatoffP\left(1+\hat{R}_\text{PA} \hat{A}\right)\hat{P} \\ 
\Dthat \hat{M} +\hat{\sigma}_0  \Dxhat \left(\hat{v} \hat{M} \right) &=\hat{D}_M \Dxhat^2 \hat{M} +\KhatonM \left(1+\hat R_\text{AM} \hat{A}\right)\left(1-\int_0^1  \hat{M}(x) \, dx\right)\\ \nonumber &- \KhatoffM\left(1+\hat{R}_\text{PM} \hat{P} \right) \hat{M},\qquad \hat{R}_\text{PM}=\frac{r_\text{PM}\Tot{P}}{\koff_M}\\
\hat{v} &= \hat{\ell}^2 \Dxhat^2 v +\hat{\ell} \Dxhat \hat{\sigma}_a(\hat{M})
\end{align}
\end{subequations}
In this model, we allow for either PAR-3 to promote myosin via the dimensionless ratio $\hat{R}_\text{AM}$, or PAR-2 to inhibit myosin via the dimensionless ratio $\hat{R}_\text{PM}$. Recalling our previous study of PAR-2 and PAR-3 in Section\ \ref{sec:P2P3}, we saw there that for strong enough mutual inhibition of the two proteins, the intrinsic bistability of PAR-3 can combine with mutual inhibition of PAR-2 to set up two mutually exclusive domains of enriched PAR-2 and PAR-3 (respectively). 

But can we use myosin as a means to shift the boundary? Figure\ \ref{fig:P2P3My} shows that indeed we can. Here we consider the same parameters as given in Table\ \ref{tab:params}, but add myosin with the parameters given in Table\ \ref{tab:paramsMy}. In order to get more bound myosin, we increase to $\kon_M=2$ $\mu$m/s (a factor of 10 relative to the value reported in \cite{gross2019guiding}, which is based on fitting a different model). The dotted lines in each plot show the steady state with $r_\text{PM}=0$, so that we can see how adding PAR-2 inhibition of myosin shifts the boundary. Starting with $r_\text{PM}=10^{-4}$ up to $r_\text{PM}=0.1$, we see a PAR-3 domain which shrinks with increasing $r_\text{PM}$, until the inhibition is so high that no flow is generated at all. 

\begin{figure}
\centering
\includegraphics[width=\textwidth]{P2P3Myosin.eps}
\caption{\label{fig:P2P3My} Fixing $r_\text{AP}=6 \times 10^{-4}$ and changing the degree to which PAR-2 inhibits myosin in the model\ \eqref{eq:PAR2PAR3My}. The dotted lines show the steady state when $r_\text{PM}=0$ (blue for PAR-3, red for PAR-2, yellow for myosin), while the solid lines show the steady state with the value of $r_\text{PM}$ indicated in the title.}
\end{figure}

What stops the boundary from expanding? 
The boundary between the domains expands when there is enough flow of PAR-2 so as to push PAR-3 from its higher steady state to its lower one. The flow of PAR-2 is linked to the gradient in myosin, which in turn depends on the gradient of PAR-2. The amount of PAR-2 comes from cytoplasm. The flow is therefore: cytoplasmic PAR-2 depleted $\rightarrow$ PAR-2 cannot increase anymore $\rightarrow$ no myosin inhibition $\rightarrow$ boundary stops moving. In the case when there is unlimited cytoplasmic PAR-2, the limit then becomes cytoplasmic PAR-3. The on rate keeps getting higher as we deplete PAR-3, until it outcompetes everything.

\subsection{Comparison with experiments}
\begin{figure}
\centering
\includegraphics[width=0.6\textwidth]{MyosinIntensityExpModel.eps}
\includegraphics[width=\textwidth]{MyosinFlowsExpModel.eps}
\caption{\label{fig:ModelExp} Comparing myosin intensity (top) and flow fields (bottom) from experiments to the model\ \eqref{eq:PAR2PAR3My} at steady state $\left(\hat t = 200\right)$. The parameters $r_\text{AP}=0.0006$ $\mu$m/s and $r_\text{PM}=0.005$ $\mu$m/s are chosen so that the A/P boundary is at 70\% of the egg length without myosin and roughly 45\% of the egg length with myosin. In all cases, we compare wild-type embryos (black) and \emph{arx-2} (RNAi) embryos (pink) with the model (red). In the flow profiles, the gray lines show individual embryos, while the black lines show the mean (error bars are one standard error in the mean).  }
\end{figure}

We now attempt an honest comparison with the experimental data. Section\ \ref{sec:RAP} already discussed how we found $r_\text{AP}$ (aPAR/pPAR inhibition strength), which leaves one unknown parameter: the inhibition of myosin by pPARs $r_\text{PM}$. We choose it so that the myosin boundary is at roughly 50\% of the embryo, thus setting $r_\text{PM}=0.005$. 

Figure\ \ref{fig:ModelExp} provides a comparison between the model and experimental data in NMY-2::GFP embryos (we are tracking myosin in fluorescence microscopy). The top plot shows the myosin intensity profile, comparing wild-type (black), branched actin knockdown via \emph{arx-2} (RNAi) (pink), and the model (red). The intensity profile in the \emph{arx-2} (RNAi) embryos compares favorably with the model; as there is a zone of low intensity in the posterior half and transitioning to zone of high intensity at the anterior half.

\fi 

\section{Coupling contractility to biochemistry }
We now present the full model with contractility and all of the biochemistry, as shown in Fig.\ \ref{fig:SchAPCM}. We will again have two protein species $A$ (representing PAR-3) and $P$ (representing PAR-2), which mutually inhibit each other. The actual inhibition of PAR-2 comes from PAR-6 and PKC-3, which colocalize with PAR-3 and are therefore not represented here. Likewise, the posterior protein which inhibits PAR-3 is actually PAR-1, which colocalizes with PAR-2 and is also therefore not represented. The mutual inhibition strengths for $A$ and $P$ were determined in Section \ref{sec:param23}, so we don't need to fit any parameters associated with these proteins.


\begin{figure}
\centering
\includegraphics[width=0.4\textwidth]{ModelSchAPCM-crop.pdf}
\caption{\label{fig:SchAPCM}Model schematic for this section. $A_1$ and $A_n$ represent PAR-3 monomers and oligomers, as well as PAR-6 and PKC-3. $P$ represents pPARs, $C$ represents CDC-42, and $M$ represents myosin.}
\end{figure}

The goal of this section is to couple the dynamics of proteins to myosin. We do this through the GTPase CDC-42, which activates myosin and branched actin. CDC-42 is typically enriched in the anterior (it is an aPAR), because it is inhibited in the posterior by its GAP, CHIN-1 \cite{sailer2015dynamic}. As shown in Fig.\ \ref{fig:SchAPCM}, we model this by introducing the species $C$, which represents CDC-42 and is inhibited by $P$, and reintroducing myosin $M$ from Section \ref{sec:myosin}. 

\subsection{Dimensionless equations}
The dimensionless equations that govern the schematic shown in Fig.\ \ref{fig:SchAPCM} are 
\begin{subequations}
\label{eq:Everything}
\begin{align}
\nonumber
&\Dthat \hat{\MA}  +\hat{\sigma}_0  \Dxhat \left(\hat{v} \hat{\MA} \right) = \DhatA \Dxhat^2 \MA +\KhatonA \left(1+\KhatfA \A{F}^+\left(\hat{\MA},\hat{\PA}\right)\right)\left(1 - \int_0^1 \hat{A}(x) \, d\hat{x} \right) \\ 
&+ 2\KhatdpA \hat{\PA}-2\KhatpA \hat{\MA}^2 - \KhatoffA \hat{\MA} \\ 
&\Dthat \hat{\PA} +\hat{\sigma}_0  \Dxhat \left(\hat{v} \hat{\PA} \right)  =\KhatpA \hat{\MA}^2- \KhatdpA \left(1+\hat{R}_\text{PA}\hat{P}\right) \hat{\PA} \\ 
&\Dthat \hat{P} +\hat{\sigma}_0  \Dxhat \left(\hat{v} \hat{P} \right)   =\DhatP \Dxhat^2 \hat{P} +\KhatonP \left(1 - \int_0^1 \hat{P}(\hat x) \, d\hat{x} \right)  - \KhatoffP \left(1+\hat{R}_\text{AP}\hat A\right) \hat{P} \\ 
\label{eq:26c}
&\Dthat \hat{C} +\hat{\sigma}_0  \Dxhat \left(\hat{v} \hat{C} \right) =\DhatC \Dxhat^2 \hat{C} +\KhatonC \left(1 - \int_0^1 \hat{C}(\hat x) \, d\hat{x} \right)  - \KhatoffC \left(1+\hat{R}_\text{PC}\hat P\right)\hat{C} \\ 
&\Dthat \hat{M} +\hat{\sigma}_0  \Dxhat \left(\hat{v} \hat{M} \right) =\hat{D}_M \Dxhat^2 \hat{M} +\KhatonM \left(1+\hat{R}_\text{CM} \hat C\right) \left(1-\int_0^1  \hat{M}(x) \, dx\right)- \KhatoffM \hat{M}\\
&\hat{v} = \hat{\ell}^2 \Dxhat^2 v +\hat{\ell} \Dxhat \hat{\sigma}_a(\hat{M})\\
\label{eq:NewNDs}
&R_\text{CM}=\frac{r_\text{CM}\Tot{C}}{\kon_M},\quad \hat{R}_\text{PC}=\frac{r_\text{PC}\Tot{P}}{\koff_C},\quad \DhatC=\frac{D_C}{L^2 \kdp_A},\quad \KhatonC=\frac{\kon_C}{\kdp_A h},\quad \KhatoffC=\frac{\koff_C}{\kdp_A}
\end{align}
\end{subequations}
These equations combine the equations \eqref{eq:MND} for myosin with the A/P cross-inhibitory circuit \eqref{eq:PAR2PAR3}, adding CDC-42 as the species $C$. In doing this, we introduce five new dimensionless parameters in\ \eqref{eq:NewNDs}. 

\subsection{Parameter estimation}
\begin{table}
\begin{small}
\centering
\begin{tabular}{|c|c|c|c|c|c|}\hline
Parameter & Description & Value & Units & Ref & Notes \\ \hline
$\C{D}$ & $C$ diffusivity & 0.1 & $\mu$m$^2$/s & \cite{robin2014single} & Known (PAR-6)\\
$\C{\koff} $ & $C$ detachment rate & 0.01 & 1/s & \cite{robin2014single}& Known (PAR-6)\\ 
$\kon_C$ & $C$ attachment rate & 0.1 & $\mu$m/s & & 50\% bound $C$ no inhibition \\ 
$\kon_M$ & $M$ attachment rate & 0.3 & $\mu$m/s & & 20\% bound $M$ no CDC \\
$\hat R_\text{PC}$ & $P$ inhibiting $C$ & 27.3  & & \cite{sailer2015dynamic} & CDC/CHIN-1 relationship (Fig.\ A5)\\ 
$\hat R_\text{CM}$ & $C$ promoting $M$ & 0.75 & &  & Correct $M$ profile\\ \hline
\end{tabular}
\caption{\label{tab:paramsC} Additional parameters and fitting parameters for coupled model shown in Fig.\ \ref{fig:SchAPCM}.}
\end{small}
\end{table}

The diffusivity and unbinding rate of PAR-6 have been measured in \cite{robin2014single}. If we assume that CDC-42 and PAR-6 have similar properties, this sets $D_C$ and $\koff_C$ as given in Table\ \ref{tab:paramsC}. There are then four unknown parameters: the on rates for CDC-42 $\kon_C$ and myosin $\kon_M$, the rate at which $P$ inhibits $C$, and the rate at which $C$ promotes $M$. We set these parameters based on the following observations. 
\begin{enumerate}
\item In \cite[Fig.~S3i]{gross2019guiding}, it is reported that roughly 25\% of PAR-6 is bound in wild-type embryos. Assuming that CDC-42 has a similar set of properties, we can assume 25\% of the protein is bound. Since the posterior half will be inhibited by CHIN-1, we look for about 50\% of the protein bound without inhibition. This sets $\kon_C$ via $\kon_C/(\kon_C+\koff_C h)=0.5$, giving $\kon_C=0.1$ $\mu$m/s. 
\item The parameter $\hat{R}_\text{PC}$ is available from the data in \cite{sailer2015dynamic}. To obtain it, we solve\ \eqref{eq:26c} at steady state in the absence of $A$ to obtain
\begin{equation}
\hat C = \frac{1}{1+\frac{h \koff_c}{\kon_C}+\frac{\hat R_\text{PC}\koff_C h}{\kon_C} \hat P}. 
\end{equation}
Now according to \cite{sailer2015dynamic}, in a system of units where $\hat C=1$ when $\hat P=0$,
\begin{equation*}
\tilde C =   \frac{1+\frac{h \koff_c}{\kon_C}}{1+\frac{h \koff_c}{\kon_C}+\frac{\hat R_\text{PC}\koff_C h}{\kon_C} \hat P}
\end{equation*} 
we have $\hat{C} \approx 1/(1+13.3\hat{P})$, which implies that 
\begin{equation*}
13.3 = \frac{\hat R_\text{PC}\koff_C h}{\kon_C\left(1+\frac{h \koff_c}{\kon_C}\right)}= \frac{\hat R_\text{PC}\koff_C h}{\kon_C+h \koff_c} \rightarrow \hat R_\text{PC}=27.3
\end{equation*}
\item In wild-type and \emph{arx-2} (RNAi) embryos, the minimum amount of bound myosin is 0.2. This sets $\kon_M$ via $\kon_M/(\kon_M+\koff_M h)=0.2$, giving $\kon_M=0.3$ $\mu$m/s.
\item We then fit the parameter $\hat R_\text{CM}=0.75$ to match the boundary position in \emph{arx-2} (RNAi) embryos, as shown in Fig.\ \ref{fig:VelProfArx2}.
\end{enumerate}

\subsection{The steady state compared to experiments}

\begin{figure}
\centering
\includegraphics[width=\textwidth]{ModelVsExpNoBA.eps}
\caption{\label{fig:StStNoBA}Steady state of the model\ \eqref{eq:Everything}, compared to experimental results. The left panel shows the steady state aPAR, pPAR, CDC-42, and myosin levels along the embryo length, with the anterior pole at $\hat x 0.5$. The middle and rightmost plots show the comparison with experimental data. We show wild-type embryos in black, and \emph{arx-2} (RNAi) embryos in pink. Results of the model (shifting the anterior pole to $\hat x = 0.25$) are ovelaid with the experimental data.}
\end{figure}

We now look at the performance of this model compared to the experimental data. We simulate the model\ \eqref{eq:Everything} (with parameters in Table\ \ref{tab:paramsC}) to steady state, and plot the results in Fig.\ \ref{fig:StStNoBA}. The left panel shows the steady state, where roughly 40\% of the domain is taken up by PAR-3. The next two panels show how the myosin intensity and flow profiles compare with the experimental data. At the correct boundary position, the flow profile that we obtain (right panel) matches with the experimental data for \emph{arx-2} (RNAi), but not wild-type embryos. Wild-type embryos tend to have a higher absolute flow speed at steady state, and the peak flow actually occurs in the \emph{posterior} domain, with the flow speed in most of the anterior domain ($\hat x = 0.25$ to $\hat x = 0.45$) being zero. The stalling of the flow in the anterior domain of wild-type embryos corresponds to a peak in the myosin intensity curve around $\hat x = 0.5$, after which we see a decrease that is not due to imaging artifacts (black curve in the middle plot).

Our conclusion from this section is that our model can successfully reproduce the dynamics in \emph{arx-2} (RNAi) embryos, but not in wild type ones. This tells us that we need to incorporate branched actin into the model to see how it might be responsible for the flow and myosin intensity curves. Of course, even for \emph{arx-2} (RNAi) embryos the model is not perfect. It gives a sharper slope in the myosin intensity curve than the experimental data, which means a smaller increase in intensity is needed to reproduce the same velocity. Nevertheless, we declare the results ``good enough'' to move on to branched actin.

\subsection{Incorporating branched actin}
Let's suppose we add branched actin to the system\ \eqref{eq:Everything}
\begin{subequations} 
\begin{align}
\tag{24e}
\label{eq:WithBA}
&\Dthat \hat{M} +\hat{\sigma}_0  \Dxhat \left(\hat{v} \hat{M} \right) =\hat{D}_M \Dxhat^2 \hat{M} +\KhatonM \left(1+\hat{R}_\text{CM} \hat C\right) \left(1-\int_0^1  \hat{M}(x) \, dx\right)- \KhatoffM \hat{M}
\end{align}
\end{subequations}


\bibliographystyle{plain}

\bibliography{../../PolarizationBib}


\end{document}
